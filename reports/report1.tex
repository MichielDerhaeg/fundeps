\documentclass[12pt]{report}
%\usepackage{a4wide}

\setlength{\parindent}{0cm}

\begin{document}
\pagestyle{myheadings}
\markright{Intermediary report November -  Student: Michiel Derhaeg}

{\bf Thesis title:} {\em Implementation of Type Inference and Elaboration into System FC for Functional Dependencies}

\vspace{0.5cm}
{\bf Supervisor:} Tom Schrijvers


\vspace{0.5cm}
{\bf Daily leaders} George Karachalias

\vspace{1cm}
{\bf Short context and goal: }
When declaring type classes with multiple parameters, it might be usefull to
define functional dependencies between these paremeters to prevent any
ambiguities during type inference. This type system feature has been implemented
in the Glasgow Haskell Compiler. But this implementation is incomplete and the
type checker rejects certain unambiguous programs which in theory should have a
principal type. Due to the inability to elaborate this into System F, an often
used intermediate representation for functional programming languages.

\paragraph{}
An extension of System F with type equality coercions has made elaboration
possible and the goal of this thesis is to implement a proposed solution to
discover practical issues that may arise and study how this solution interacts
with other major type system features like GADTs and Type Families.

\vspace{1cm}
{\bf Literature study:}
\begin{itemize} % TODO which authors need to be included? which journal?
\item D. Vytiniotis, S. P. Jones, T. Schrijvers, M. Sulzmann; "OutsideIn(X);
    Modular type inference with local assumptions". TODO, 2017
\item M. Sulzmann, M. M. T. Chakravarty, S. P. Jones, K. Donnelly; "System F
    With Type Equality Coercions". TODO, 2011
\item G. Butto, G. Karachalias, T. Schrijvers; "Quantified Class Constraints".
    ACM SIGPLAN, 2017
\item G. Karachalias, T. Schrijvers; "Elaboration on Functional Dependencies".
    ACM SIGPLAN, 2017
\item P. Wadler, S. Blott; "How to make ad-hoc polymorphism less ad hoc". TODO, 1988
\end{itemize}

\vspace{1cm}
{\bf Finished work:}
Firstly, I have studied the literature concerning type systems. I've read the
book \textit{Types and Programming Languages} and several papers about type
systems, System $F_C$ and type classes. I've cleaned up a prototype compiler
that was used for a previous project and implented a simpler type class
constraint solver so it just supports Haskell '98.


\vspace{1cm}
{\bf Main results:}
? % TODO not sure what do say here


\vspace{1cm}
{\bf Main difficulties:}
The main unexpected time investment was moving from the theory to the
implementation. Figuring out which parts of the code coincided with wich parts
of the theory took an unexpected amount of effort initially. But once this
hurdle had passed and I was familiarized with the implementation, this was no
longer a issue.

\vspace{1cm}
{\bf Planning:}
? % TODO if only i had those notes

\vspace{1cm}
{\bf Continuing to work like this, I expect to achive a ??/20 by the end of the
thesis.}

{\bf I expect to hand in my thesis in June.}


\end{document}
