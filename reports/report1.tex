\documentclass[12pt]{report}
%\usepackage{a4wide}

\setlength{\parindent}{0cm}

\begin{document}
\pagestyle{myheadings}
\markright{Intermediary report November -  Student: Michiel Derhaeg}

{\bf Thesis title:} {\em Implementation of Type Inference and Elaboration into System $F_C$ for Functional Dependencies}

\vspace{0.5cm}
{\bf Supervisor:} Tom Schrijvers


\vspace{0.5cm}
{\bf Daily leaders:} George Karachalias

\vspace{1cm}
{\bf Short context and goal: }
When declaring type classes with multiple parameters, it is often useful to
define functional dependencies between these paremeters to prevent any
ambiguities during type inference. This type system feature has been implemented
in the Glasgow Haskell Compiler, but, the implementation is incomplete and the
type checker rejects certain unambiguous programs which in theory should have a
principal type.

\mbox{}\\
An extension of System F with type equality coercions has made elaboration
possible and the goal of this thesis is to implement a proposed solution to
discover practical issues that may arise and study how this solution interacts
with other major type system features like GADTs and Type Families.

\vspace{1cm}
{\bf Literature study:}
\begin{itemize}
\item
  D. Vytiniotis, S. Peyton jones, T. Schrijvers, and M. Sulzmann. 2011.
  OutsideIn(X): Modular Type Inference with Local Assumptions. J. Funct.
  Program. 21, vol. 4-5, pp. 333--412.
\item
  M. Sulzmann, M. M. T. Chakravarty, S. Peyton Jones, and K. Donnelly. 2007.
  System F with Type Equality Coercions.
  In \textit{TLDI '07}. ACM.
\item
  G.J. Bottu, G. Karachalias, T. Schrijvers, B. C. d. S. Oliveira, and P. Wadler. 2017.
  Quantified Class Constraints.
  In \textit{Haskell 2017}. ACM.
\item
  G. Karachalias and T. Schrijvers. 2017.
  Elaboration on Functional Dependencies. %: Functional Dependencies Are Dead, Long Live Functional Dependencies!.
  In \textit{Haskell 2017}. ACM.
\item
  P. Wadler and S. Blott. 1989.
  How to Make Ad-hoc Polymorphism Less Ad Hoc.
  In \textit{POPL '89}. ACM.
\end{itemize}

\vspace{1cm}
{\bf Finished work:}
Firstly, I have studied some related work concerning type systems. I have read
the book \textit{Types and Programming Languages} and several papers concerning
Haskell's type system, System $F_C$, type classes, and functional dependencies.

Furthermore, I have simplified the implementation of a prototype compiler for a
Haskell-like functional language, which has already been developed by the
research group.

% George says: 'cleaned up' is too informal for this sort of text
% George says: Use "I have" instead "I've"

\vspace{1cm}
{\bf Main results:}
We now have an implementation of a prototype compiler for a basic functional
programming language with type classes which we can easily extend with the
features we will later require.

Additionally, the implementation currently supports superclass constraints
without relying on backtracking (as opposed to the non-simplified version of
the compiler), which is essential for implementing type inference for
functional dependencies.

\vspace{1cm}
{\bf Main difficulties:}
A significant amount of time had to be invested into familiarizing myself with
the research area, but this was to be expected, as I had little to no prior
knowlegde about it.

An unexpected amount of time has been invested in moving from the theory to the
implementation. Figuring out the correlation between the code and the theory
took an unexpected amount of effort initially.

We've opted for a type class constraint solver without non-determinism. We
therefore had to devise another method to deal with superclasses.

\vspace{1cm}
{\bf Planning:}
% TODO changeme, i'll put something here already
By the end of November I expect to have extended the target language of the
compiler, which currently is System F,  with type equality coercions. And have
finished the parts of the thesis text concerning the required background.
% better word than background

\vspace{1cm}
{\bf Continuing to work like this, I expect to achive a ??/20 by the end of the
thesis.}

{\bf I expect to hand in my thesis in June.}


\end{document}
