\documentclass[12pt]{report}
%\usepackage{a4wide}

\setlength{\parindent}{0cm}

\begin{document}
\pagestyle{myheadings}
\markright{Intermediary report March -  Student: Michiel Derhaeg}

{\bf Thesis title:} {\em Implementation of Functional Dependencies}

\vspace{0.5cm}
{\bf Supervisor:} Tom Schrijvers


\vspace{0.5cm}
{\bf Daily leaders:} George Karachalias

\vspace{.5cm}
{\bf Short context and goal: }
When declaring type classes with multiple parameters, it is often useful to
define functional dependencies between these paremeters to prevent any
ambiguities during type inference. This type system feature has been implemented
in the Glasgow Haskell Compiler, but, the implementation is incomplete and the
type checker rejects certain unambiguous programs which in theory should have a
principal type.

\mbox{}\\
An extension of System F with type equality coercions has made elaboration
possible and the goal of this thesis is to implement a proposed solution to
discover practical issues that may arise, and study how this solution interacts
with other major type system features like GADTs and Type Families.

\vspace{.5cm}
{\bf Literature study:}
\begin{itemize}
\item
  D. Vytiniotis, S. Peyton jones, T. Schrijvers, and M. Sulzmann. 2011.
  OutsideIn(X): Modular Type Inference with Local Assumptions. J. Funct.
  Program. 21, vol. 4-5, pp. 333--412.
\item
  M. Sulzmann, M. M. T. Chakravarty, S. Peyton Jones, and K. Donnelly. 2007.
  System F with Type Equality Coercions.
  In \textit{TLDI '07}. ACM.
\item
  G.J. Bottu, G. Karachalias, T. Schrijvers, B. C. d. S. Oliveira, and P. Wadler. 2017.
  Quantified Class Constraints.
  In \textit{Haskell 2017}. ACM.
\item
  G. Karachalias and T. Schrijvers. 2017.
  Elaboration on Functional Dependencies. %: Functional Dependencies Are Dead, Long Live Functional Dependencies!.
  In \textit{Haskell 2017}. ACM.
\item
  P. Wadler and S. Blott. 1989.
  How to Make Ad-hoc Polymorphism Less Ad Hoc.
  In \textit{POPL '89}. ACM.
\item
  M. Sulzmann, G. J. Duck, S. Peyton Jones and P. J. Stuckey. 2006.
  Understanding functional dependencies via constraint handling rules.
  J. Funct. Program. 1, vol. 17, pp. 83--129.
\item
  M. P. Jones. 2000.
  Type classes with functional dependencies.
  In \textit{ESOP/ETA (LNCS) 2000}. pp. 230--244.
\item
  M. M. T. Chakravarty, G. Keller, S. Peyton Jones and S. Marlow. 2005.
  Associated Types with Class.
  In \textit{POPL '05}. ACM.
\item
  M. M. T. Chakravarty, G. Keller and S. Peyton Jones. 2005.
  Associated Type Synonyms.
  In \textit{ICPF '05}. ACM.
\end{itemize}

\vspace{.5cm}
{\bf Finished work:}
Firstly, I have read several papers concerning functional dependencies, GADTs,
and type families.
%
Secondly, the specification of Karachalias and Schrijvers (2017) omitted
several cases, type signatures, all mentions of kinds and contained small
errors. I have addressed such omissions and errors.
%
Finally, I have extended the prototype compiler with support for
multi-parameter type classes and functional dependencies. The new type checker
still needs to be thoroughly tested.

\vspace{.5cm}
{\bf Main results:}
We now have an implementation of a prototype compiler for a basic functional
programming language with type classes which is extended with support for
functional dependencies.
%
Because the inference algorithm uses type families internally, it can very
easily be extended to support user-defined type families.
%
The specification mentions the so-called {\em type-level projection functions},
which are assumed to exist for all data types. I have discovered that this
approach is not as expressive as full support for kind polymorphism, and thus
undermines the expressiveness of the system of Karachalias and Schrijvers
(2017).

\vspace{.5cm}
{\bf Main difficulties:}
Because kind polymorphism appears to be the best solution for the projection
issue, we have opted to implement it. This implies that there probably will not
be enough time to implement GADTs.

\vspace{.5cm}
{\bf Planning:}
% pun intended. Nice pun mate! ^^
I expect the implementation to be fully functional and tested by next week.
After that, the implementation will be extended with source-level type families
so that their interaction with functional dependencies can be studied.

\vspace{.5cm}
{\bf Continuing to work like this, I expect to achieve a 16/20 by the end of the
thesis.}
%
{\bf I expect to hand in my thesis in June.}

\end{document}
