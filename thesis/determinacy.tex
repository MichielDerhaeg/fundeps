\section{Determinacy}
\label{sec:determinacy}
The determinacy relation is one of the key idea's of the formalization of
functional dependencies used in the work of Karachalias et
al.~\cite{Karachalias:2017:EFD:3156695.3122966}.  It takes the form of
$det(\overline{a},\overline{\pi}) = \theta$ and can be interpreted as follows:
It produces a substitution that replaces the type variables of $\overline{\pi}$
with entirely equivalent types that only use type variables from $\overline{a}$.
It does so by leveraging the functional dependencies of the type class
constraints in question. Every functional dependency gives rise to a type family
that maps the domain to the range of the dependency, allowing us to
\textit{determine} the range using the domain, hence the name
\textit{determinacy}. Consider this simple example:

\[
\begin{array}{l c l}
    \texttt{class C a b | a -> b} &\rightsquigarrow& \textbf{type} \; F_C(a) \\
    \texttt{instance C Int Bool} &\rightsquigarrow& \textbf{axiom} \; g :
    F_C(Int) \sim Bool \\
    \\
\end{array}
\]
\[
    det(c, C \; c \; d) = [d \mapsto F_C (c)]
\]

The functional dependency of the class \texttt{C} gives rise to a type family
$F_C$ which can be used to refer to the type variable $d$ using the type
variable $c$. The instance declaration gives rise to an axiom $g$ that states
that $F_C(Int)$ is equal to $Bool$. This could be used to guide type inference, if
the inference algorithm decides that that $a$ is equal to $Int$, it can use the
axiom $g$ to determine that $F_C(c)$, and indirectly $d$, should be equal to
$Bool$. A second use of this relation is to verify if type variables can be
determined by the given type variables and class constraints. One simply needs
to inspect the domain of the substitution for an occurence of this type
variable.

\begin{figure}
\fbox{$\overline{a}; \overline{\pi} \vdash_{D} \theta \rightsquigarrow \theta'$}
\begin{mathpar}
\inferrule*[right=$\text{Step}_D$]
{
    TC \; \overline{u} \in \overline {\pi}
    \\
    TC \; \overline{a} \mid a_{i_1}, \mathellipsis, a_{i_n} \rightarrow a_{i_0}
    \\
    fv(u_{i_0}) \nsubseteq \overline{a} \cup dom(\theta)
    \\
    fv(\overline{u}^{i_n}) \subseteq \overline{a} \cup dom(\theta)
}
{
    \overline{a}; \overline{\pi} \vdash_{D} \theta \rightsquigarrow
    project(u_{i_0},F_{TC_i}(\theta(\overline{u}^{i_n}))) \cdot \theta
}
\end{mathpar}
\caption{Determinacy Relation}
\label{fig:determinacy}
\end{figure}

The determinacy relation $det(\overline{a},\overline{\pi}) = \theta$ is formally
defined as $\overline{a};\overline{\pi} \vdash_{D} \bullet \rightsquigarrow^!
\theta$ where the single determinacy step $\overline{a};\overline{\pi}
\vdash_{D} \theta \rightsquigarrow \theta'$ is defined in
Figure~\ref{fig:determinacy}. The exclamation mark $!$ indicates that we
repeatedly apply the single determinacy step until no additional substitution is
produced for any type class constraint.

For every type class constraint, for every instantiated functional dependency,
if the range of the dependency has not yet been determined but the domain can be
determined, we update the substitution to map the range to the type family
application of the types in the domain of the dependency. Consider the following
example:
\[
\begin{array}{l}
\texttt{class G a b | a -> b}\\
\texttt{class H a b | a -> b}\\
\end{array}
\]

Computing the determinacy substitution $det(a, \{G \; a \; c, H \; c \; b \} )$
would go as follows:
\[
\begin{array}{l l c l}
a; \{\highlight{G \; a \; c}, H \; c \; b\} &\vdash_{D} \bullet
&\rightsquigarrow& [c \mapsto F_G(a)]
\\
a; \{G \; a \; c, \highlight{H \; c \; b}\} &\vdash_{D}  [c \mapsto F_G(a)]
&\rightsquigarrow& [b \mapsto F_H(F_G(a)), c \mapsto F_G(a)]
\\
a; \{G \; a \; c, H \; c \; b\} &\vdash_{D}  [b \mapsto F_H(F_G(a)), c \mapsto
F_G(a)] &\not\rightsquigarrow&
\end{array}
\]

In the first iteration, only the $G \; a \; c$ matches, as $c$ can be determined
by $a$. The $[c \mapsto F_G(a)]$ substitution would then allow the $H \; c \; b$
constraint to be matched, as $c$ is now indirectly determined by $a$ which
allows $b$ to be determined by $c$. All type variables of every range of every
functional dependency are now accounted for and the determinacy relation halts.

Not every set of type class constraints would contain just type variables like
in the previous example. This is what the $project$ function specified in
Figure~\ref{fig:project} is for. To illustrate what it does consider the
following:

$det(a, C \; a \; (Either \;
String \; c))$ can not have $[c \mapsto F_C(a)]$ as a result, $a$ determines
$Either$ $String$ $c$ and not just $c$. If we had a type family $EitherR$
available with the following axiom
\[
  \textbf{axiom} \; eitherR \; a_1 \; a_2 : EitherR(Either \; a_1 \; a_2) \sim a_2
\]
determinacy could instead correctly produce $[c \mapsto EitherR(F_C(a))]$. We
generalize this to $Proj^T_i$, a projection type family for every type
constructor $T$ and parameter with index $i$. These need to be generated for
every data type declaration, for every parameter of the type constructor.

\[
\begin{array}{l l}
  \textbf{type} \; &Proj^T_i(a)
  \\
  \textbf{axiom} \; g \; \overline{a}^n : &Proj^T_i(T \; a_1 \mathellipsis a_n)
  \sim a_i
\end{array}
\]

To really drive the point home, the following more complicated example $det(a, C
\; a \; (Either \; (Maybe \; b) \; c) )$ would produce the following
substitution:
\[
[ b \mapsto Proj^{Maybe}_1(Proj^{Either}_1(F_C(a))), c \mapsto
Proj^{Either}_2(F_C(a))]
\]

\begin{figure}
\fbox{$project(u, \tau_{fam}) = \theta$}
\[
\begin{array}{l c l}
    project(a, \tau) & = & [a \mapsto \tau]
    \\
    project(T \; \overline{u}, \tau) & = &
    \overline{project(u_i, Proj^T_i(\tau))}
    \\
    project(T, \tau) & = & \bullet
\end{array}
\]
\caption{Type Constructor Projection Function}
\label{fig:project}
\end{figure}

Notice that $project$ is a partial function, it only considers type applications
of the form $T \; \overline{u}$ where the leftmost component is a type
constructor. Unfortunately this is not always the case. Consider the following
slightly contrived example:

\begin{verbatim}
    class A a b | a -> b
    class B a b | a -> b
    class C a b | a -> b
    instance (Functor f, A a (f b), B b c) => C a c
\end{verbatim}

The type variable $c$ in the instance declaration of \texttt{C} can be
determined indirectly through $b$. But $b$ would need to be projected out of $(f
\; b)$. We can't use a projection of the form $Proj^T_i$ because $f$ is not a
type constructor but a variable so in this case we would require something more
general.

If our Haskell-like system and our version of \systemfc were to be extended
with kind polymorphism~\cite{yorgey}, we could instead use a version of the
$project$ function that would be more practical and support these cases. This
version of $project$ is specified in Figure~\ref{fig:poly-project}. Instead of
defining projection type families for every type constructor we would only have
to generate exactly two. $L$ and $R$ project the left and right component
respectively of a type application.
%TODO example?

\[
\begin{array}{l c l}
    \textbf{type} \; L &: &\forall k_1 \; k_2. (a : k_1) \rightarrow k_2
    \\
    \textbf{type} \; R &: &\forall k_1 \; k_2. (a : k_1) \rightarrow k_2
    \\
    \\
    \textbf{axiom} \; proj_L &:& L((u_1 : k_2 \rightarrow k_1) \; (u_2 : k_2))
    \sim u_1
    \\
    \textbf{axiom} \; proj_R &:& R((u_1 : k_2 \rightarrow k_1) \; (u_2 : k_2))
    \sim u_2
\end{array}
\]
\begin{figure}
\fbox{$project(u, \tau_{fam}) = \theta$}
\[
\begin{array}{l c l}

    project(a, \tau) & = & [a \mapsto \tau]
    \\
    project(u_1 \; u_2, \tau) & = &
    project(u_1, L(\tau)) \cdot project(u_2, R(\tau))
    \\
    project(T, \tau) & = & \bullet
\end{array}
\]
\caption{Poly-kinded Projection Function}
\label{fig:poly-project}
\end{figure}
