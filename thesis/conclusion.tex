\chapter{Conclusion}
\label{cha:conclusion}

TODO i guess talk about results
\begin{itemize}
    \item successful implementation of the spec
    \item spec of elaboration of evidence of $OutsideIn(X)$
    \item validation of the spec?
\end{itemize}

\section{Future work}

The implementation presented in this work consists of approximately 8300 lines
of Haskell code. Considering that GHC consists of 90000 of Haskell code and
65000 of C code at the time of writing, it is safe to say that the
implementation presented in this work should be easier to understand and modify
due to it's relative simplicity. This property makes it an ideal testbed for
extensions of the language.

\subsection{Skolemization}

Consider the example from Section~\ref{sec:functional-dependencies}:
\begin{verbatim}
    class Collection c e | c -> e where
      singleton :: e -> e
    singleton2 e = singleton (singleton e)
\end{verbatim}

Without an explicit type signature, the type that would be inferred for
\texttt{singleton2} would be:

\begin{verbatim}
    singleton2 :: (Collection F(a) F(F(a)), Collection a F(a))
               => F(F(a)) -> a
\end{verbatim}

This is perfectly fine in theory but if this is shown to the user upon request
or as part of an error it might cause some confusion if the user is not familiar
with the internal workings.

We propose to develop a skolemization procedure that would transform the type
signature into an equivalent signature devoid of any FD-induced type families.
We present a simple sketch of this procedure that would perhaps remind the
reader of \textit{flattening} presented in Section~\ref{sec:canonicalization}:

\[
\begin{array}{l c l}
    & & (Collection \; \highlight{F(a)} \; F(\highlight{F(a)}), Collection \; a
    \; \highlight{F(a)}) \Rightarrow F(\highlight{F(a)}) \rightarrow a \\

    [b \mapsto F(a)] &\rightsquigarrow& (Collection \; b \; \highlight{F(b)}, Collection \;
    a \; b) \Rightarrow \highlight{F(b)} \rightarrow a \\

    [c \mapsto F(b)] &\rightsquigarrow& (Collection \; b \; c, Collection \; a
    \; b) \Rightarrow c \rightarrow a
\end{array}
\]

We search for an occurrence of a type family application that contains no nested
type families and substitute this type family application with a newly generated
type variable. We repeat this operation until no type families can be found.
For the \texttt{Collection} example we would first replace $F(a)$ with the newly
generated $b$. During the second iteration we would replace $F(b)$ with $c$.

%TODO can i steal a name like this?
%alternative: Named Functional Dependencies
\subsection{Putting a Name to Functional Dependencies}

Mark P. Jones suggested in his original work about functional
dependencies~\cite{Jones00typeclasses} to add the option for the programmer to
annotate functional dependencies with a label. Consider the following example:

\begin{verbatim}
    class          C a b | f :: a -> b
    class C a b => D a b | f :: a -> b
\end{verbatim}

Jones' original intention was to allow the programmer to indicate that class
declaration of \texttt{D} inherits the functional dependency labeled by
\texttt{f} from the superclass \texttt{C}. However, in modern formalizations
this 'inheritance' is inferred automatically and the idea was not explored any
further.

Inspired by this idea and because the dependencies are represented internally as
type families, these labels could be used to expose them to the user. Instead of
writing \texttt{g :: C a b => a -> b} the user could refer to the dependency as
the type-level functions they actually are and instead write \texttt{g :: C a (f
a) => a -> f a}. Because we are merely exposing the internals to the user, this
should be straightforward to implement. Therefore it could be interesting to
investigate how useful this can be.

\subsection{Interaction with Modern Features}

The language we defined in this thesis is a subset of the dialect of Haskell
that is currently being used at the time of writing. Many newer type system
features have been added to GHC and other implementations since the inception of
functional dependencies in the year 2000. If this formalization is to be
integrated in industry standard implementations like GHC, it should first be
investigated how it integrates alongside these newer features and how they
interact.

\subsubsection{Type Families}

Given that the formalization used in this work uses type families internally, we
expect that extending the implementation to support them in the source text
should be straightforward. The constraint solver is already well-equipped to
deal with them. Other components often assume that they don't appear
in the source text and will need to be adapted, in particular the determinacy
relation, discussed in Section~\ref{sec:determinacy}, and the conditions,
discussed Chapter~\ref{cha:conditions}.

Because type families and functional dependencies would then share their
underlying framework, it would be an excellent opportunity to study and
improve the interaction between the two features. Current implementations of
functional dependencies have been proven to be difficult to integrate with type
families due to the differences of their current formalizations. As a consequence,
their interaction remains poorly understood to this day.

%TODO include example? useful? it is funny though
Consider the following example:

\begin{verbatim}
    class C a b | a -> b
    type family F x

    instance C a (F [b]) => C a [b]
\end{verbatim}

\begin{verbatim}
    type Wat t = F t
    instance C a (Wat [b]) => C a [b]
\end{verbatim}

\subsubsection{GADTs}

Because $OutsideIn(X)$ has been explicitly designed to support GADTs, we expect
this extension would also be straightforward to implement despite the fact that
we omitted GADT support in the $OutsideIn(X)$ implementation present prototype
compiler. We speculate that integrating this with functional dependencies is
possible without too many complications.
