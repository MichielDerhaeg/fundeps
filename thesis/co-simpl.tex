\chapter{Coercion Simplification}
\label{cha:co-simpl}

Every step the constraint entailment algorithm takes, a new piece of evidence is
emitted that mirrors this action. It often produces enormous coercions for even
the simplest of programs. This makes the resulting program more difficult to
parse for humans and the coercions themselves almost impossible to comprehend.
What makes matters even worse is that every time FD-induced equalities are not
required, these coercions would simply prove $a \sim a$ for any type $a$. They
are essentially very complicated reflections, making them, and the type cast
that uses them, entirely redundant.

\begin{figure}
\fbox{$\Gamma_{co} \vdash_{s} t \rightsquigarrow t'$}
\[
\begin{array}{l c l}
\Gamma_{co} \vdash_s t \triangleright \langle \tau \rangle &\rightsquigarrow& t\\
\\
\Gamma_{co} \vdash_s \gamma &\rightsquigarrow& \langle \tau \rangle\\
\where \Gamma_{co} \vdash_{co} \tau \sim \tau
\\
\Gamma_{co} \vdash_s \langle \tau \rangle \fctrans \gamma &\rightsquigarrow& \gamma
\\
\Gamma_{co} \vdash_s \gamma \fctrans \langle \tau \rangle &\rightsquigarrow& \gamma
\\
\Gamma_{co} \vdash_s \langle \tau_1 \rangle \; \langle \tau_2 \rangle
&\rightsquigarrow& \langle \tau_1 \; \tau_2 \rangle
\\
\Gamma_{co} \vdash_s \text{sym} \; (\text{sym} \; \gamma)) &\rightsquigarrow& \gamma
\\
\Gamma_{co} \vdash_s \text{sym} \; (\langle \tau \rangle) &\rightsquigarrow& \langle \tau \rangle
\\
%TODO this one is done in the oppisite direction in the fc paper
\Gamma_{co} \vdash_s \text{sym} \; \gamma_1 \fctrans \text{sym} ; \gamma_2
&\rightsquigarrow& \text{sym} \; (\gamma_1 \fctrans \gamma_2)
\\
\Gamma_{co} \vdash_s F(\overline{\gamma}) \fctrans F(\overline{\gamma}')
&\rightsquigarrow& F(\overline{\gamma_i \fctrans \gamma_i'})
\end{array}
\]
\caption{Coercion Simplification Rules}
\label{fig:co-simpl}
\end{figure}

% and in an attempt to preserve the author's sanity.
To alleviate this problem, we employ a coercion simplification procedure that
transforms these coercions to ones that are strictly smaller but entirely
equivalent. This procedure is inspired by the coercion normalization procedure
by Sulzmann et al.~\cite{Sulzmann:2007:SFT:1190315.1190324}. Like
normalization, it consists of several rewrite rules specified in
figure~\ref{fig:co-simpl}. We call it simplification instead of normalization
because we employ a much smaller set of rules that only simplifies the coercions
instead of reducing it to some normal form. These rules have proven to be highly
effective and sufficient so no effort has been made to implement full
normalization.

However, this is trivial given that all of the surrounding framework is in
place. This includes support for type-aware rewrite rules, these are rules that
take the type (the proposition) of a coercion into account. For example:
\[
\begin{array}{l}
\Gamma_{co} \vdash_s \gamma \rightsquigarrow \langle \tau \rangle\\
\where \Gamma_{co} \vdash_{co} \tau \sim \tau
\end{array}
\]
This rule checks if the coercion $\gamma$ proves an equality between two
syntactic equivalent types. This indicates that this coercion can be replaced by
a reflection. In order to be able to perform coercion typing we need to
do something similar to what is specified in Figure~\ref{fig:fc-co} from
Chapter~\ref{cha:system-fc}. However, we would like to keep coercion
simplification a separate operation, completely independent from \systemfc ~type
checking. Therefore, we present a simplified form of coercion typing, specified
in Figure~\ref{fig:fc-co-type} in the appendix. This version does not verify
well-formedness of any kind and only performs the necessary operations to the
derive the proposition from the coercion.

The simplified coercion typing makes use of a simplified typing environment,
specified in Figure~\ref{fig:co-env} and denoted by $\Gamma_{co}$. As type
checking types is not required and determining syntactic equality of types or
observing their structure does not require any context or environment, it
suffices to only keep track of coercion variables brought into scope and axiom
variables defined by axiom declarations.

\begin{figure}
\[
\Gamma_{co} ::= \bullet \mid \Gamma_{co}, \; c : \tau_1 \sim \tau_2 \mid
\Gamma_{co}, \; g \; \overline{a} : F(\overline{v}) \sim v
\]
\caption{Coercion Environment}
\label{fig:co-env}
\end{figure}
