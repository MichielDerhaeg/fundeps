\chapter{Coercion Simplification}
\label{cha:co-simpl}

Every step the constraint entailment algorithm takes, a new piece of evidence is
emitted that mirrors this action. It often produces enormous coercions for even
the simplest of programs. This makes the resulting program more difficult the
parse for humans and the coercions themselves almost impossible to comprehend.
What makes matters even worse is that every time FD induced equalities are not
required ,these coercions would simply prove $a \sim a$ for any type $a$. They
are essentially very complicated reflections, making them, and the type cast that
uses them, completely useless.

% and in an attempt to preserve the author's sanity.
To alleviate this problem, we employ a coercion simplification procedure that
transforms these coercions to ones that are strictly smaller but entirely
equivalent.

\begin{figure}
%TODO define environment
\[
\begin{array}{l c l}
simplCo(\gamma) \\
\guard \Gamma \vdash_{co} \rightsquigarrow \tau \sim \tau &=& \langle \tau
\rangle
\\
simplCo(\langle \tau \rangle \fctrans \gamma) &=& \gamma
\\
simplCo(\gamma \fctrans \langle \tau \rangle) &=& \gamma
\\
simplCo(\langle \tau_1 \rangle \; \langle \tau_2 \rangle) &=& \langle \tau_1 \;
\tau_2 \rangle
\\
simplCo(\text{sym} \; (\text{sym} \; \gamma)) &=& \gamma
\\
simplCo(\text{sym} \; (\langle \tau \rangle)) &=& \langle \tau \rangle
\\
simplCo(\text{sym} \; \gamma_1 \fctrans \text{sym} ; \gamma_2) &=& \text{sym} \;
(\gamma_1 \fctrans \gamma_2)
\\
simplCo(F(\overline{\gamma}) \fctrans F(\overline{\gamma}')) &=&
F(\overline{\gamma_i \fctrans \gamma_i'})
\end{array}
\]
\caption{Coercion Simplification}
\label{fig:co-simpl}
\end{figure}
