% vim:ft=tex
\section{Introduction}
Wadler and Blot \cite{Wadler:1989:MAP:75277.75283} introduced type classes as a
way to express ad-hoc polymorphism and became extremely popular and an essential
part of the Haskell programming language. This feature was quickly extended to
allow for multiple type parameters. Type classes originally allowed us to
express properties of a collection of types and now to relations between types
as well. By usage of these multi-parameter type classes situations could easily
arise where the type of one or several parameters could not be determined
unambiguously. To resolve these amibuities Mark P. Jones intruduced the notion
of Functional Dependencies \cite{Jones00typeclasses} between the type parameters
of type classes so one would be uniquely determined by the other and the
amibuities would be resolved. It's a technique that originated from relation
database theory but applied to the design of type systems. It became a very
popular type system feature that is still used very often to date. Not just to
resolve amibuities, but also for the static enforcement of semantic properties
and type-level programming.

Implemented in GHC % TODO

Proved to be problematic. Some programs that are entirely correct would be
rejected by GHC's type checker because at the time it seemd to be impossible to
translate the typing relationship into System F. A polymorphic lambda calculus
often used by functional programming language compilers as an intermediate
representation. No implementation uses pure System F to encode their source
language in. It is usually extended to make the translation more straightforward
, readable or simply to support more advanced features. E.g. it was extended to
support GADTs. But Functional Dependencies did not seem to be worth it.

To support features like Associated Types and Type Families, it was proposed by
Sulzmann et al. \cite{Sulzmann:2007:SFT:1190315.1190324} to extend System F with
type equality coercions. Which is now called System $F_C$ and currently used by
GHC as the new intermediate representation.

Despite the continued popularity of Functional Dependencies it was never
investigated if it could be implemented using the new representation until
recently. Karachalias and Schrijvers \cite{Karachalias:2017:EFD:3122955.3122966}
described a type inference algorithm for functional dependencies and a way to
translate them into System $F_C$
