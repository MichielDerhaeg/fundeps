% vim:ft=tex
\chapter{Introduction}
\label{cha:intro}
Type classes were introduced by Wadler and Blott~\cite{Wadler:1989:MAP:75277.75283}
as a
way to express ad-hoc polymorphism, and they have become an essential part of the Haskell
programming language. This feature was quickly extended to allow for multiple
type parameters. Type classes originally allowed us to express properties of a
collection of types and now relations between types as well. Unfortunately, by using
multi-parameter type classes, situations can easily arise where the type
of one or several parameters can not be determined unambiguously. To address this issue,
Mark P. Jones introduced the notion of Functional Dependencies
\cite{Jones00typeclasses} between the type parameters of type classes, so one
would be uniquely determined by the other and the ambiguities would be resolved.
This feature originates from relational database theory, but---as Jones has shown---can also be applied to
the design of type systems. It became a very popular type system feature that is
used to date. Not just to resolve ambiguities, but also for the
static enforcement of semantic properties and type-level programming.

However, the implementation of this feature in GHC proved to be problematic and
does not behave according to the specification. Programs that are correct according
to specifications of functional dependencies
have been rejected by GHC's type checker for years, since at the time of their development is was not known how to translate them into System
F~\cite{systemf},
a polymorphic
lambda calculus often used by functional programming language compilers as an
intermediate representation. No language compiler uses pure System F to encode
their source language in. It is usually extended to make the translation more
straightforward, readable, or simply to support more advanced features. For
example, algebraic datatypes are a common addition. In GHC specifically, the intermediate language was replaced in 2007
by System $F_C$~\cite{Sulzmann:2007:SFT:1190315.1190324}, an extension of
System F with type equality coercions and open, non-parametric type-level
functions. This allowed for the accommodation of advanced features such as
Generalized Algebraic Data Types (GADTs)~\cite{PeytonJones06} and type
families~\cite{AssociatedTypeSynonyms}.

Though Functional Dependencies remain a popular feature, it was never
investigated whether they can be translated into System $F_C$, until
recently. Karachalias and Schrijvers \cite{Karachalias:2017:EFD:3156695.3122966}
described a type inference algorithm for functional dependencies and a way to
translate them into System $F_C$.

