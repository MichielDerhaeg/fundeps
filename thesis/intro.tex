% vim:ft=tex
\chapter{Introduction}
\label{cha:intro}
Type classes were introduced by Wadler and Blott~\cite{Wadler:1989:MAP:75277.75283}
as a
way to express ad-hoc polymorphism, and they have become an essential part of the Haskell
programming language. This feature was quickly extended to allow for multiple
type parameters. Type classes originally allowed us to express properties of a
collection of types and now relations between types as well. Unfortunately, by using
multi-parameter type classes, situations can easily arise where the type
of one or several parameters can not be determined unambiguously. To resolve
these ambiguities, Mark P. Jones introduced the notion of Functional Dependencies
\cite{Jones00typeclasses} between the type parameters of type classes, so one
would be uniquely determined by the other and the ambiguities would be resolved.
It is a technique that originated from relational database theory but applied to
the design of type systems. It became a very popular type system feature that is
used to date. Not just to resolve ambiguities, but also for the
static enforcement of semantic properties and type-level programming.

However, the implementation of this feature in GHC proved to be problematic and
does not behave according to the specification.  Some programs that are entirely
correct
would be rejected by GHC's type checker because at the time it seemed to be
impossible to translate the typing relation into System
F~\cite{systemf},
a polymorphic
lambda calculus often used by functional programming language compilers as an
intermediate representation. No implementation uses pure System F to encode
their source language. It is usually extended to make the translation more
straightforward, readable or simply to support more advanced features. For
example, algebraic datatypes are a common addition. In GHC specifically, the intermediate language has been extended
with support for GADTs~\cite{PeytonJones06}, which seemed over-sophisticated extension to an
intermediate representation at the time. Motivated by
keeping the intermediate language as simple as possible, it was never extended
to support Functional Dependencies and its implementation remained incomplete.

To support features like type
families~\cite{AssociatedTypeSynonyms}, it was proposed by
Sulzmann et al. \cite{Sulzmann:2007:SFT:1190315.1190324} to extend System F with
type equality coercions. This calculus is called System $F_C$ and currently used by
GHC as the new intermediate representation. This new extension is expressive
enough to express most of the more advanced type system features.

Despite the continued popularity of Functional Dependencies it was never
investigated if they could be implemented using the new representation until
recently. Karachalias and Schrijvers \cite{Karachalias:2017:EFD:3156695.3122966}
described a type inference algorithm for functional dependencies and a way to
translate them into System $F_C$.
