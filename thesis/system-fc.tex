\chapter{System $F_C$}
\label{cha:system-fc}
%TODO subst $F_C$ with $\text{F}_\text{C}$ everywhere

Most modern compiler implementations make use of intermediate representations
(\textit{IRs}), languages used to encode the intermediary results of the
various compilation phases. These are used for various reasons, the IR could be
very similar a assembly language so the compiler can easily target several
different architectures as only the last phases are required to differentiate
between the different target architectures. It could be similar to the source
language but simpler so even though the source language is extended regularly
this doesn't have to mean that the rest of the compiler has to change as long as
we can translate these extensions to the this IR.

System F and it's variants often assume the role of the latter for functional
programming language implementations. These are (explicitly) typed IRs, meaning
that they are the result of type inference and all type system features of the
source language need to be translated to and encoded in the type system of the
intermediate language. In a perfect world this would not be necessary, but
unfortunately this is not the case and people, including compiler developers,
make mistakes and create bugs. Being able to type check the IR catches many of
these mistakes which has proven to be incredibly useful. The author of this work
has discovered the usefulness of this first-hand while implementing the
algorithms discussed in the later chapters.

In this chapter we'll define and discuss a variant of System $F_C$, the new
target language of the compiler. It is extended from System F in a few ways that
makes it able to support a multitude of more advanced source-language type
system features like type families, GADTs, and most relevant to us, Functional
Dependencies.

System $F_C$ extends polymorphic lambda calculus with explicit type equality
witnesses, or proofs, called coercions. And with type-level functions and
top-level axioms which are their instantiations.

\section{Syntax}

\begin{figure}
\small
\[
\begin{array}{l l r}
    v &::= a \;  \; | \;  \; T \;  \; | \;  \; v_1 \;  \; | \;  \; v_2  \; | \;
    \highlight{F(\overline{v})}  \; | \;  \forall a. v  \; | \; \highlight{\psi
    \Rightarrow v} &\textit{Types}\\ \\
    \highlight{\gamma} &::= \highlight{\langle v \rangle \; | \; \text{sym} \; \gamma \; | \;
    \text{left} \; \gamma \; | \; \text{right} \; \gamma \; | \; \gamma_1
    \fctrans
    \gamma_2 \; | \; \psi \Rightarrow \gamma} &\textit{Coercions} \\ 
    \; &\quad \highlight{| \; \; F(\overline{\gamma}) \; | \; \forall a . \; \gamma \; | \;
    \gamma_1 [\gamma_2] \; | \; g \; \overline{v} \; | \; c \; | \; \gamma_1 @
    \gamma_2 \; | \; \gamma_1 \; \gamma_2} \\ \\
    \highlight{\psi} &::= \highlight{v_1 \sim v_2} &\textit{Propositions}\\ \\
    t &::= x \; | \; K \; | \; \Lambda a . \; t \; | \; t \; v \; | \;
    \lambda(x\; : \; v). \; t \; | \; t_1 \; t_2 \; | \; \highlight{\Lambda(c \; : \psi).
    \; t} &\textit{Terms} \\
          &\quad | \; \; \highlight{t \; \gamma} \; | \; \highlight{t \triangleright \gamma} \; | \;
    \textbf{case} \; t_1 \; \textbf{of} \; \overline{p \rightarrow t_2} \; | \;
    \textbf{let} \; x : v = t_1 \; \textbf{in} \; t_2 \\ \\
    p &::= K \; \highlight{(\overline{b : k})} \; \highlight{(\overline{c : \psi})} \;
    (\overline{x \highlight{: v}}) &\textit{Patterns}\\ \\
    decl &::= \textbf{data} \; T \; \overline{a} \; \textbf{where} \; K : v
         &\textit{Declarations}\\
         &\quad | \; \; \textbf{let} \; x : v = t \\
         &\quad | \; \; \highlight{\textbf{type} \; F(\overline{a})} \\
         &\quad | \; \; \highlight{\textbf{axiom} \; g \; \overline{a} :
         F(\overline{u}) \sim \; v} \\ \\
    pgm &::= \overline{decl} &Program \\ \\
    \Gamma &::= \bullet \mid \Gamma, \; a : k \mid \Gamma, \; x : v \mid \Gamma,
    \; \highlight{c : \gamma} \mid \Gamma, \; \highlight{g \; \overline{a} :
    F(\overline{v}) \sim v} &Typing \; Environment
\end{array}
\]
\caption{System $F_C$ syntax}
\label{fc-syntax}
\end{figure}

The highlighted parts are the components of the language which are extensions of
standard System F with data types.

\subsection{Types}

Unlike Haskell and other Hindley-Milner based systems, we do not discriminate
between polytypes, qualified types or monotypes. System $F_C$, like System F, is
impredicative and types like $(\forall a. \; a \rightarrow a) \rightarrow
(\forall a. \; a \rightarrow a)$ are allowed. Type abstractions and
quantification can appear nested in the type. We denote type \textit{variables}
with $a$, type \textit{constructors} with $T$. Type \textit{abstractions} with
$\forall a. \; v$, \textit{application} with $v_1 \; v_2$. New to System $F_C$
are \textit{type family applications}, denoted by $F(\overline{v})$, and
\textit{qualified} types, denoted by $\phi \Rightarrow v$. Unlike with type
applications and type constructors, type family applications are defined with
all of their parameters. Type families are always required to be fully
saturated, 'partial' type family applications are not allowed.

The typing rules for type well-formedness are specified in
Figure~\ref{fig:fc-type}

\subsection{Coercions}

Coercions, denoted by $\gamma$, are the main interesting addition to the target
language. These are proofs which are usually constructed during the elaboration
of the source language. These coercions can then be used during type checking to
explicitly cast a term $e : v_1$ to type $v_2$ if a coercion is provided that
witnesses the equality between these types. Their semantics are completely
determined by their typing as they do not have any runtime semantics.

$\langle v \rangle$ captures reflexivity which proves $v \sim v$ for any
type $v$. $\text{sym} \; \gamma$ captures symmetry, this reverses the
orientation of the type equality. $\text{left} \; \gamma$ and $\text{right} \;
\gamma$ captures projection of type applications.
$\gamma_1 \fctrans \gamma_2$ is transitivity, combining $v_1 \sim v_2$ and $v_2
\sim v_3$ into $v_1 \sim v_3$. $\psi \Rightarrow \gamma$ are qualified
coercions. $F(\overline{\gamma})$ and $\gamma_1 \; \gamma_2$ represent type
(family) application for coercions. $\gamma_1[\gamma_2]$ and $\gamma_1@\gamma_2$
are type and coercion applications respectively. And lastly, $\forall a. \;
\gamma$ is type abstraction for coercions.

Coercions make use of two symbols, $g$ are axiom variables used to refer to the
equality defined in axiom declarations. $c$ are coercion variables, introduced
locally by pattern matching or by coercion abstractions. To accomodate for the
newly introduced variables, the typing environment $\Gamma$ has been extended to
keep track of these variables. The typing rules for coercions are specified in
Figure~\ref{fig:fc-co}.

\subsection{Propositions}

Propositions, denoted by $\psi$, could be informally described as the types of
coercions. They are the type equality proven by the coercion like how types
classify terms and kinds classify types. Well-formednes is specified in
Figure~\ref{fig:fc-prop}, this simply consists of well-formedness of both types
of the equality.

\subsection{Terms}

Terms have three notable extensions. Coercion abstraction, denoted by $\Lambda(c
: \psi). \; t$, coercion application, denoted by $t \; \gamma$, and explicit
type casts, denoted by $t \triangleright \gamma$. The typing rules are specified
in Figure~\ref{fig:fc-term}. The first two allow us to explicitly pass coercions
and are quite straightforward.

The most noteworthy are the type casts. Given a coercion $\gamma : v_1 \sim v_2$
and a term $t$ of type $v_1$ we can explicitly type cast the type of $t$ to
$v_2$ using $\gamma$, the proof that $v_1$ is equal to $v_2$.  This is why we
often claim that System $F_C$ has support for non-syntactic type equality. We
can use a term of type $v_1$ where we expect a term of a syntactically distinct
type $v_2$ as long as we can prove that $v_1$ is equal to $v_2$ by other means.

\subsection{Patterns}

Patterns can now introduce existential type variables and local equality
constraints, elevating them to the level of expressiveness of GADTs. Matching on
these patterns brings these existential type variables $\overline{b}$ annotated
with their kinds $\overline{k}$ into scope. The local type equalities, denoted
by $\overline{\phi}$, are made accessible by bringing their accompanying
coercion variables $\overline{c}$ into scope as well. So they might be used by
explicit type casts later on. Lastly, note the subtle addition of explicit type
annotations $\overline{v}$ to the term variables $\overline{x}$. This is done to
make the type checking as simple as possible at the cost of making elaboration
slightly more tedious. The typing of patterns is specified in rule \texttt{Alt}
in Figure~\ref{fig:fc-term} for pattern match case alternatives.

\subsection{Declarations}

GADT syntax is used for data type declarations, data constructors are defined by
their annotated type signature instead of the types of their individual
parameters like how algebraic data types would be defined in Haskell.

%TODO mention somewhere that type families are *open* type functions
Type family declarations are quite straightforward, they simply define the name
of the declared family as well as the arity of it's parameters and their kinds
and the kind of the result.

%TODO don't want to mention type patterns, not really enforced anywhere in the
%implementation
Equality axioms, as the name suggests, axiomatically define type equalities
between type family applications $F(\overline{v})$ and some other type $v$. As
mentioned earlier, type casts allow for the use of non-syntactic equalities.
Axioms on the other hand provide a way of introducing these non-syntactic
equalities. Axioms are annotated with a axiom variable $g$ so we can refer to
these axioms in coercions. The type variables $\overline{a}$ in the declaration
indicate that axioms can be used for any instantiation for $\overline{a}$, or in
other words, they are polymorphic.

Top-level value bindings are entirely standard.

%TODO declaration typing

\section{Type checking}

System $F_C$ is by design very straightforward to type check, no type inference
is required of any kind and can be done by a single traversal of the program.
One of the properties that contributes to this is that System $F_C$ is
syntax-directed, there is a single typing rule for every syntactic component.
Every rule would simply correspond to a single function in the implementation
that type checks this component and simply mirrors what is specified in the
typing rules.

Note that in Figure~\ref{fig:fc-term} rule \texttt{TmCase}, the type of the case
scrutinee would be passed to the typing rule of \texttt{Alt} as an input.
Meaning that $\overline{v}_a$ should be an input for the implementation of the
\texttt{Alt} rule.

\section{Elaboration}

The next example illustrates how the broken example from
Section~\ref{sec:problem-statement} can be correctly elaborated into System
$F_C$. Everything that appears after a $\rightsquigarrow$ represents the
generated System $F_C$.

\begin{verbatim}
    class C a b | a -> b
\end{verbatim}

$\qquad\rightsquigarrow$ \textbf{type} $FC$ $a$

$\qquad\rightsquigarrow$ \textbf{data} $CDict$ $a$ $b$ \textbf{where}

$\qquad\qquad$ $CDict$ $::$ $(FC \; a \sim b)
\Rightarrow$ $CDict$ $a$ $b$

\begin{verbatim}
    instance C Int Bool
\end{verbatim}

$\qquad\rightsquigarrow$ \textbf{axiom} $g$ : $FC$ $Int$ $\sim$ $Bool$

$\qquad\rightsquigarrow$ \textbf{let} $d$ : $CDict$ $Int$ $Bool$

$\qquad\qquad\quad$ = $CDict$ $Int$ $Bool$ $(g : FC$ $Int \sim Bool)$

\begin{verbatim}
    f :: C Int b => b -> Bool
    f x = x
\end{verbatim}

$\qquad\rightsquigarrow$ let $f$ = $\Lambda b. \; \lambda (d : CDict \; Int \;
b). \;\lambda (x : b). \; \text{case} \; d \; \text{of}$

$\hspace{3.5cm} CDict \; (c : FC \; Int \sim b ) \rightarrow x \triangleright
(\text{sym} \; c) \fctrans g$
\\

Type classes are not a part of the \systemfc syntax and therefore need to be
encoded somehow. They are translated to so called ``dictionaries''.  These are
values that contain the functions of the method implementations of the instance
declaration. These dictionaries are explicitly passed around as parameters to
functions with type class constraints so their methods can be available to them
at runtime. In \systemfc however, these dictionaries could be used to store
coercions as well.

Every functional dependency of a type class declaration gives rise to a type
family declaration for a type family that maps the domain of the dependency to
the range.  The functional dependency in the class declaration of \texttt{C}
will generate a type family declaration ``\textbf{type}~$FC$~$a$'' to represent
the functional dependency.

The class declaration itself will generate a data declaration for the dictionary
with a single data constructor to store functional dependency related type
equalities and method implementations. It should also store the dictionaries of
the super classes and any existential type variables introduced by those, this
will be explained further in Section~\ref{sec:class-elaboration} about class
elaboration. A coercion that proves the equality of $FC$ $a$ and $b$, denoted by
$FC$ $a \sim b$ in the type signature, should be stored by the dictionary
$CDict$ generated for the type class \texttt{C}.

The instance declaration for \texttt{C Int Bool} generates a top-level axiom
$g$ which states that $FC$ $Int$ is indeed equal to $Bool$. The instance
declaration also generates a dictionary value that would contain the method
implementations of the instance, if it had any, and the top
level axiom as a coercion constant.

In the elaboration of the function \texttt{f}, the polymorphic type \texttt{b}
becomes an explicit type parameter. This originates from plain System $F$ and it
is how it supports parametric polymorphism. The type class constraint becomes a
normal term parameter that holds the corresponding dictionary. And lastly, the
parameter \texttt{x} of type \texttt{b}.

The function body of $f$ pattern matches against the dictionary to expose the
coercion within, which is used by the compiler to construct a coercion to
explicitly cast $x$ from \texttt{b} to \texttt{Bool}

The ``sym'' in the coercion changes the orientation of the equality and the
``$\fctrans$'' witnesses transitivity of two coercions. Here we annotate every
subcomponent of the coercion with it's proposition to make it clear how the
final proposition is derived:

\[
\quad\quad((\text{sym} \; (c : FC \; Int \sim b)) : b \sim FC \; Int) ; (g : FC
\; Int \sim Bool) ) : b \sim Bool
\]

This means that the coercion proves the equality of $b$ and $Bool$ and
\texttt{x} can be cast safely using a type cast $\triangleright$ and the
generated coercion.

% TODO other typing rules, declarations, etc
\begin{figure}[h]
\fbox{$\Gamma \vdash_{tm} t : v$}
% TERM
$$
%TmCast
\inferrule*[right=TmCast]
{P;\Gamma \vdash_{tm} t : v_1 \\ P;\Gamma \vdash_{co} \gamma : v_1 \sim v_2}
{P;\Gamma \vdash_{tm} t \triangleright \gamma : v_2}
$$
$$
%TmVar
\inferrule*[right=TmVar]
{x : v \in \Gamma}
{\Gamma \vdash_{tm} x : v}
$$
$$
%TmCon
\inferrule*[right=TmCon]
{K : v \in \Gamma}
{\Gamma \vdash_{tm} K : v}
$$
$$
%TmAbs
\inferrule*[right=TmAbs]
{x \notin dom(\Gamma) \qquad \Gamma \vdash_{ty} v_1 : \star \qquad \Gamma , x : v_1
\vdash_{tm} t : v_2}
{\Gamma \vdash_{tm} \lambda(x : v_1). t : v_1 \rightarrow v_2}
$$
$$
%TmApp
\inferrule*[right=TmApp]
{\Gamma \vdash_{tm} t_1 : v_2 \rightarrow v_1 \qquad \Gamma \vdash_{tm} t_2 : v_2 }
{\Gamma \vdash_{tm} t_1 \; t_2 : v_1}
$$
$$
%TmTyApp
\inferrule*[right=TmTyApp]
{\Gamma \vdash_{tm} t : \forall (a : k). v \qquad \Gamma \vdash_{ty} v_1 : k}
{\Gamma \vdash_{tm} t \; v_1 : [a \mapsto v_1]v}
$$
$$
%TmTyAbs
\inferrule*[right=TmTyAbs]
{a \notin dom(\Gamma) \\ \Gamma , a : k \vdash_{tm} t : v}
{\Gamma \vdash_{tm} \Lambda(a : k). t: \forall(a : k). v}
$$
$$
%TmCoAbs
\inferrule*[right=TmCoAbs]
{
    c \notin dom(\Gamma)
    \\
    \Gamma \vdash_{pr} \psi
    \\
    \Gamma, c : \psi \vdash_{tm} t : v
}
{
    \Gamma \vdash_{tm} \Lambda(c : \psi). \; t : \psi \Rightarrow v
}
$$
$$
%TmCoApp
\inferrule*[right=TmCoApp]
{
    \Gamma \vdash_{tm} t : \psi \Rightarrow v
    \\
    \Gamma \vdash_{co} \gamma : \psi
}
{
    \Gamma \vdash_{tm} t \; \gamma : v
}
$$
$$
%TmLet
\inferrule*[right=TmLet]
{
    x \notin dom(\Gamma)
    \\
    \Gamma, x : v_1 \vdash_{tm} t_1 : v_1
    \\
    \Gamma \vdash_{ty} v_1 : \star
    \\
    \Gamma, x : v_1 \vdash_{tm} t_2 : v_2
}
{
    \Gamma \vdash_{tm} (\textbf{let} \; x : v_1 = t_1 \; \textbf{in} \; t_2) : v_2
}
$$
$$
%TmCase
% TODO mention v_scr is input for alt rule
\inferrule*[right=TmCase]
{
    \Gamma \vdash_{tm} t_{scr} : v_{scr}
    \\
    \overline{\Gamma \vdash_{alt} p \rightarrow t : v_{scr} \rightarrow v}
}
{
    \Gamma \vdash_{tm} \textbf{case} \; t_{scr} \; \textbf{of} \; \overline{p
    \rightarrow t} : v
}
$$
$$
%Alt
\inferrule*[right=Alt]
{
    (K : \forall (\overline{a : k}) (\overline{b' : k'}) . \; \overline{\psi} \Rightarrow
    \overline{v} \rightarrow T \; \overline{a}) \in \Gamma
    \\
    \theta = [\overline{a} \mapsto \overline{v}_a, \overline{b'} \mapsto
    \overline{b}]
    \\
    \overline{b}, \overline{c}, \overline{x} \notin dom(\Gamma)
    \\
    \Gamma, (\overline{b : k'}), (\overline{c : \theta(\psi)}), (\overline{x :
    \theta(v)}) \vdash_{tm} t : v_2
}
{
    \Gamma \vdash_{alt} K \; (\overline{b : k'}) \; (\overline{c : \theta(\psi)})
    \; (\overline{x : \theta(v)}) \rightarrow t : T \; \overline{v}_a
    \rightarrow v_2
}
$$
\caption{System $F_C$ Term Typing}
\label{fig:fc-term}
\end{figure}

\begin{figure}
\fbox{$\Gamma \vdash_{ty} v : k$}
% TYPES
\begin{mathpar}
%TyVar
\inferrule*[right=Tyvar]
{a : k \in \Gamma}
{\Gamma \vdash_{ty} a : k}
\\
%TyQual TODO prop psi?
\inferrule*[right=TyQual]
{\Gamma \vdash_{pr} \psi \qquad \Gamma \vdash_{ty} v : k}
{\Gamma \vdash_{ty} \psi \Rightarrow v : k}
\\
%TyCon
\inferrule*[right=TyCon]
{T : k \in \Gamma}
{\Gamma \vdash_{ty} T : k}
\\
%TyApp
\inferrule*[right=TyApp]
{\Gamma \vdash_{ty} v_1 : k1 \rightarrow k_2 \qquad \Gamma \vdash_{ty} v_2 : k_1}
{\Gamma \vdash_{ty} v_1 v_2 : k_2}
\\
%TyAll
\inferrule*[right=TyAll]
{\Gamma , a \vdash_{ty} v : \star \qquad a \notin fv(\Gamma)}
{\Gamma \vdash_{ty} \forall a . v : \star}
\\
%TyFam TODO no theory(P) in my fc juist environment \Gamma ?
\inferrule*[right=TyFam]
{F(\overline{k}) : k' \in \Gamma \qquad \Gamma \vdash_{ty} v_i : k_i}
{\Gamma \vdash_{ty} F(\overline{v}) : k'}
\end{mathpar}
\caption{System $F_C$ Type Well-formedness}
\label{fig:fc-type}
\end{figure}
\begin{figure}
\fbox{$\Gamma \vdash_{pr} \psi$}
$$
%Prop
\inferrule*[right=Prop]
{\Gamma \vdash_{ty} v_1 : k \qquad \Gamma \vdash_{ty} v_2 : k}
{\Gamma \vdash_{pr} v_1 \sim v_2}
$$
\label{fig:fc-prop}
\end{figure}
\begin{figure}
\fbox{$\Gamma \vdash_{co} \gamma : \psi$}
\begin{mathpar}
%CoVar
\inferrule*[right=CoVar]
{
    (c : \psi) \in \Gamma
}
{
    \Gamma \vdash_{co} c : \psi
}
\\
%CoAx
\inferrule*[right=CoAx]
{
    (g \; (\overline{a : k}) : v_1 \sim v_2) \in \Gamma
    \\ 
    \overline{\Gamma \vdash_{ty} v : k}
}
{
    \Gamma \vdash_{co} g \; \overline{v} : [\overline{a} \mapsto \overline{v}]v_1
    \sim [\overline{a} \mapsto \overline{v}]v_2
}
\\
%CoRefl
\inferrule*[right=CoRefl]
{
    \Gamma \vdash_{ty} v : k
}
{
    \Gamma \vdash_{co} \langle v \rangle : v \sim v
}
\\
%CoSym
\inferrule*[right=CoSym]
{
    \Gamma \vdash_{co} \gamma : v_1 \sim v_2
}
{
    \Gamma \vdash_{co} \text{sym} \; \gamma : v_2 \sim v_1
}
\\
%CoTrans
\inferrule*[right=CoTrans]
{
    \Gamma \vdash_{co} \gamma_1 : v_1 \sim v_2
    \\
    \Gamma \vdash_{co} \gamma_2 : v_2 \sim v_3
}
{
    \Gamma \vdash_{co} \gamma_1 \fctrans \gamma_2 : v_1 \sim v_3
}
\\
%CoApp
\inferrule*[right=CoApp]
{
    \Gamma \vdash_{ty} v_1 \; v_3 : k
    \\
    \Gamma \vdash_{co} \gamma_1 : v_1 \sim v_2
    \\
    \Gamma \vdash_{co} \gamma_2 : v_3 \sim v_4
}
{
    \Gamma \vdash_{co} \gamma_1 \; \gamma_2 : v_1 \; v_3 \sim v_2 \; v_4
}
\\
%CoLeft
\inferrule*[right=CoLeft]
{
    \Gamma \vdash_{co} \gamma : v_1 \; v_2 \sim v_3 \; v_4
}
{
    \Gamma \vdash_{co} \text{left} \; \gamma : v_1 \sim v_3
}
\\
%CoRight
\inferrule*[right=CoRight]
{
    \Gamma \vdash_{co} \gamma : v_1 \; v_2 \sim v_3 \; v_4
}
{
    \Gamma \vdash_{co} \text{right} \; \gamma : v_2 \sim v_4
}
\\
%CoFam
\inferrule*[right=CoFam]
{
    (F : \overline{k} \rightarrow k') \in \Gamma
    \\
    \overline{\Gamma \vdash_{co} \gamma : v_1 \sim v_2}
    \\
    \overline{\Gamma \vdash_{ty} v_1 : k}
}
{
    \Gamma \vdash_{co} F(\overline{\gamma}) : F(\overline{v_1}) \sim
    F(\overline{v_2})
}
\\
%CoAbs
\inferrule*[right=CoAbs]
{
    \Gamma, a : k \vdash_{co} \gamma : v_1 \sim v_2
    \\
    \Gamma, a : k \vdash_{ty} v_1 : k_1
    \\
    (a : k) \notin dom(\Gamma)
}
{
    \Gamma \vdash_{co} \forall a. \; \gamma : \forall. a \; v_1 \sim \forall. a
    \; v_2
}
\\
%CoInst
\inferrule*[right=CoInst]
{
    \Gamma \vdash_{co} \forall a. \; \gamma_1 : \forall. a \; v_1 \sim \forall. a
    \; v_2
    \\
    \Gamma \vdash_{co} \gamma_2 : v_3 \sim v_4
    \\
    \Gamma \vdash_{ty} v3 : k
}
{
    \Gamma \vdash_{co} \gamma_1 [\gamma_2] : [a \mapsto v_3]v_1 \sim [a \mapsto
    v_4]v_2
}
\\
%CoQual
\inferrule*[right=CoQual]
{
    \Gamma \vdash_{pr} \psi
    \\
    \Gamma \vdash_{co} \gamma : v_1 \sim v_2
}
{
    \Gamma \vdash_{co} \psi \Rightarrow \gamma : (\psi \Rightarrow v_1) \sim
    (\psi \Rightarrow v_2)
}
\\
%CoQual
\inferrule*[right=CoQInst]
{
    \Gamma \vdash_{co} \gamma_1 : (\psi \Rightarrow v_1) \sim
    (\psi \Rightarrow v_2)
    \\
    \Gamma \vdash_{co} \gamma_2 : \psi
}
{
    \Gamma \vdash_{co} \gamma_1 @ \gamma_2 : v_1 \sim v_2
}
\end{mathpar}
\caption{System $F_C$ Coercion Typing}
\label{fig:fc-co}
\end{figure}


\section{Operational Semantics}
%TODO seems irrelevant, don't care about this
%TODO what about meta-theory and such?
