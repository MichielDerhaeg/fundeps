\section{System $F_C$}

In this section we'll define and discuss System $F_C$, the new target language
of the compiler. It is extended from System F in a few ways that makes it able
to support a multitude of more advanced source-language type system features
like Type Families, Generalised Algebraic Datatypes, and most relevant to us,
Functional Dependencies.

System $F_C$ extends polymorphic lambda calculus with explicit type equality
witnesses, or proofs, called coercions. And with type-level functions and
top-level axioms which are their instantiations.

% Discuss other additions other additions to System F like datatypes

\subsection{Syntax}

%TODO highlight new stuff
\begin{figure}
\begin{align*}
    v &::= a \;  \; | \;  \; T \;  \; | \;  \; v_1 \;  \; | \;  \; v_2  \; | \;
    \highlight{F(\overline{v})}  \; | \;  \forall a. v  \; | \; \highlight{\psi
    \Rightarrow v} &\textit{Types}\\ \\
    \highlight{\gamma} &::= \highlight{\langle v \rangle \; | \; \text{sym} \; \gamma \; | \;
    \text{left} \; \gamma \; | \; \text{right} \; \gamma \; | \; \gamma_1 ;
    \gamma_2 \; | \; \psi \Rightarrow \gamma} &\textit{Coercions} \\ 
    \; &\quad \highlight{| \; \; F(\overline{\gamma}) \; | \; \forall a . \; \gamma \; | \;
    \gamma_1 [\gamma_2] \; | \; g \; \overline{v} \; | \; c \; | \; \gamma_1 @
    \gamma_2 \; | \; \gamma_1 \; \gamma_2} \\ \\
    \highlight{\psi} &::= \highlight{v_1 \sim v_2} &\textit{Propositions}\\ \\
    t &::= x \; | \; K \; | \; \Lambda a . \; t \; | \; t \; v \; | \;
    \lambda(x\; : \; v). \; t \; | \; t_1 \; t_2 \; | \; \highlight{\Lambda(c \; : \psi).
    \; t} &\textit{Terms} \\
          &\quad | \; \; t \; \gamma \; | \; \highlight{t \triangleright \gamma} \; | \;
    \textbf{case} \; t_1 \; \textbf{of} \; p \rightarrow t_2 \; | \;
    \textbf{let} \; x : v = t_1 \; \textbf{in} \; t_2 \\ \\
    p &::= K \; \highlight{\overline{b} \; (\overline{c : \psi})} \; (\overline{d : \tau})
    \; (\overline{f : v}) &\textit{Patterns}\\ \\
    decl &::= \textbf{data} \; T \; \overline{a} \; \textbf{where} \; K : v
         &\textit{Declarations}\\
         &\quad | \; \; \textbf{let} \; x : v = t \\
         &\quad | \; \; \highlight{\textbf{type} \; F(\overline{a})} \\
         &\quad | \; \; \highlight{\textbf{axiom} \; g \; \overline{a} :
         F(\overline{u}) \sim \; v} \\
\end{align*}
\caption{System $F_C$ syntax}
\label{fc-syntax}
\end{figure}

The highlighted parts are the things that are different from standard System F.

\subsubsection{Type equality coercions}

Coercions, denoted by $\gamma$, are the main interesting addition to the target
language. These are proofs which are usually constructed during the elaboration
of the source language. These coercions can then be used during typechecking to
explicitly cast a term $e : v_1$ to type $v_2$ if a coercion is provided that
witnesses the equality between these types.

\subsubsection{Type functions}

\subsection{Typing}
% TODO other typing rules
\begin{figure}[h]
% TERM
$$
%TmCast
\inferrule
{P;\Gamma \vdash_{tm} t : v_1 \\ P;\Gamma \vdash_{co} \gamma : v_1 \sim v_2}
{P;\Gamma \vdash_{tm} t \triangleright \gamma : v_2}
\; \textsc{TmCast}
$$
$$
%TmVar
\inferrule
{x : v \in \Gamma}
{\Gamma \vdash_{tm} x : v}
\; \textsc{TmVar}
$$
$$
%TmCon
\inferrule
{K : v \in \Gamma}
{\Gamma \vdash_{tm} K : v}
\; \textsc{TmCon}
$$
$$
%TmAbs
\inferrule
{x \notin \Gamma \qquad \Gamma \vdash_{ty} v_1 : \star \qquad \Gamma , x : v_1
\vdash_{tm} t : v_2}
{\Gamma \vdash_{tm} \lambda(x : v_1). t : v_1 \rightarrow v_2}
\; \textsc{TmAbs}
$$
$$
%TmApp
\inferrule
{\Gamma \vdash_{tm} t_1 : v_2 \rightarrow v_1 \qquad \Gamma \vdash_{tm} t_2 : v_2 }
{\Gamma \vdash_{tm} t_1 \; t_2 : v_1}
\; \textsc{TmApp}
$$
$$
%TmTyApp
\inferrule
{\Gamma \vdash_{tm} t : \forall (a : k). v \qquad \Gamma \vdash_{ty} v_1 : k}
{\Gamma \vdash_{tm} t \; v_1 : [a \mapsto v_1]v}
\; \textsc{TmTyApp}
$$
$$
%TmTyAbs
\inferrule
{a : k \notin \Gamma \qquad \Gamma , a : k \vdash_{tm} t : v}
{\Gamma \vdash_{tm} \Lambda(a : k). t: \forall(a : k). v}
\; \textsc{TmTyAbs}
$$
% TYPES
$$
%TyVar
\inferrule
{a : k \in \Gamma}
{\Gamma \vdash_{ty} a : k}
\; \textsc{TyVar}
$$
$$
%TyQual TODO prop psi?
\inferrule
{\Gamma \vdash_{pr} \psi \qquad \Gamma \vdash_{ty} v : k}
{\Gamma \vdash_{ty} \psi \Rightarrow v : k}
\; \textsc{TyQual}
$$
$$
%TyCon
\inferrule
{T : k \in \Gamma}
{\Gamma \vdash_{ty} T : k}
\; \textsc{TyCon}
$$
$$
%TyApp
\inferrule
{\Gamma \vdash_{ty} v_1 : k1 \rightarrow k_2 \qquad \Gamma \vdash_{ty} v_2 : k_1}
{\Gamma \vdash_{ty} v_1 v_2 : k_2}
\; \textsc{TyApp}
$$
$$
%TyAll
\inferrule
{\Gamma , a \vdash_{ty} v : \star \qquad a \notin fv(\Gamma)}
{\Gamma \vdash_{ty} \forall a . v : \star}
\; \textsc{TyAll}
$$
$$
%TyFam TODO no theory(P) in my fc juist environment \Gamma ?
\inferrule
{F(\overline{k}) : k' \in \Gamma \qquad \Gamma \vdash_{ty} v_i : k_i}
{\Gamma \vdash_{ty} F(\overline{v}) : k'}
\; \textsc{TyFam}
$$
$$
%Prop
\inferrule
{\Gamma \vdash_{ty} v_1 : k \qquad \Gamma \vdash_{ty} v_2 : k}
{\Gamma \vdash_{pr} v_1 \sim v_2}
\; \textsc{Prop}
$$
$$
%CoAx
\inferrule
{g (\overline{a : k}) : v_1 \sim v_2 \in \Gamma \qquad \overline{\Gamma
\vdash_{ty} v : k}}
{\Gamma \vdash_{co} g \; \overline{v} : [\overline{a} \mapsto \overline{v}]v_1
\sim [\overline{a} \mapsto \overline{v}]v_2}
\; \textsc{CoAx}
$$
\caption{System $F_C$ typing rules}
\label{fc-type}
\end{figure}


\subsection{Operational Semantics}

% why systemfc?
% syntax directed => easy to type check
% should be easy to construct
% powerful enough to express many things
