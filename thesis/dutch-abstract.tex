\textit{Type classes} zijn een van de meest populaire functies van Haskell en
hebben van wijdverspreid gebruik genoten sinds hun ontwerp door Wadler en Blott.
Niet lang na hun introductie in Haskell-compilers, werden \textit{type classes}
met één parameter uitgebreid met ondersteuning voor meerdere typeparameters.
Waardoor de expressieve kracht van \textit{class constraints} ging van simpele
predikaten op types tot de relaties daartussen.

Helaas kunnen er zich gemakkelijk situaties voordoen door het gebruik van
\textit{type classes} met meerdere parameters waarbij het type van één of
meerdere van de parameters niet ondubbelzinnig kan worden bepaald. Om dit
probleem aan te pakken introduceerde Mark P. Jones het concept van
\textit{Functional Dependencies} tussen de typeparameters van \textit{type
classes}, opdat de één uniek zou kunnen worden bepaald door de ander and de
ambiguiteiten opgelost zouden zijn.  \textit{Functional dependencies} zijn zeer
populair geworden en worden tot op de dag van vandaag gebruikt, niet gewoon om
ambiguiteiten op te lossen, maar ook voor het statisch afdwingen van semantische
eigenschappen en om te programmeren met types.

Helaas, ondanks de populariteit van deze \textit{functional dependencies},
blijft deze component van de taal onbetrouwbaar geïmplementeerd in de primaire
Haskell-compiler, GHC. De oorzaak van dit probleem is de niet-parametrische aard
van dit concept, wat niet kan worden vertaald naar simpele parametrische
polymorfisme.

Desondanks is de tussenrepresentatie van GHC sindsdien aanzienlijk veranderd
sinds de introductie van \textit{functional dependencies} in het jaar 2000 om te
voorzien voor een verscheidendheid van type-gerelateerde functionaliteit.
Namelijk in 2007 was de tussentaal van GHC vervangen door \systemfc, een
uitbreiding van System F met expliciete bewijzen van gelijkheid tussen types,
die we \textit{coercions} noemen. De mogelijkheid om \textit{functional
dependencies} te implementeren door ze te vertalen in deze meer expressieve
calculus is tot voor kort niet volledig onderzocht. In 2017 presenteerden
Karachalias en Schrijvers \textit{type inference} algoritme voor
\textit{functional dependencies} en een manier om werk ze uit in \systemfc.

Deze thesis legt een implementatie voor van het bovengenoemde algoritme en een
gedetailleerd overzicht van belangrijke implementatiespecifieke details en
complicaties die zich hebben voorgedaan tijdens de implementatie van
\textit{functional dependencies}.
%
Het doel van deze implementatie is tweevoudig.
%
Ten eerste dient het als een validatie en evaluatie van de specificatie van
Karachalias en Schrijvers. Inderdaad, onze implementatie heeft er verschillende
onthuld onnauwkeurigheden in de specificatie van het algoritme; we laten zien
hoe ze kunnen worden aangepakt.
%
Ten tweede dient het als een haalbaarheidsstudie, gericht op het kwantificeren
van de uitdagingen dat kan zich voordoen bij het integreren van functionele
afhankelijkheden in de stroom Haskell ecosysteem. We laten zien dat de benodigde
veranderingen minimaal zijn door het type inferentieraamwerk dat momenteel in
GHC aanwezig is.
