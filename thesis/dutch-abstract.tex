\textit{Type classes} zijn een van de populairste functies van Haskell en hebben
van wijdverspreid gebruik genoten sinds hun ontwerp door Wadler en Blott.  Niet
lang na hun introductie in Haskell-compilers werden \textit{type classes} met
één parameter uitgebreid, waardoor ze ook meerdere typeparameters ondersteunen.
Hierdoor werden \textit{class constraints} gepromoveerd van simpele predikaten
op types tot de relaties daartussen.

Helaas kunnen er zich door het gebruik van \textit{type classes} met meerdere
parameters gemakkelijk situaties voordoen waarbij het type van één of meerdere
van de parameters niet ondubbelzinnig kan worden bepaald. Om dit probleem aan te
pakken, introduceerde Mark P. Jones het concept van \textit{Functional
Dependencies} tussen de typeparameters van \textit{type classes}, opdat de één
uniek zou kunnen worden bepaald door de ander en de ambiguïteiten opgelost
zouden zijn. \textit{Functional dependencies} zijn zeer populair geworden en
worden tot op de dag van vandaag gebruikt. Ze worden niet alleen gebruikt om
ambiguïteiten op te lossen, maar ook om semantische eigenschappen statisch af te
dwingen en met types te programmeren.

Tot op heden blijft deze component van de taal, ondanks de populariteit van deze
\textit{functional dependencies}, nog altijd onbetrouwbaar geïmplementeerd in de
primaire Haskell-compiler, GHC. De oorzaak van dit probleem is de
niet-parametrische aard van het concept, die niet kan worden vertaald naar
simpele parametrische polymorfisme.

Toch is de tussenrepresentatie van GHC aanzienlijk veranderd sinds
\textit{functional dependencies} zijn geintroduceerd in 2000 om te accommoderen
voor een verscheidenheid van type-geralateerde functionaliteit. In 2007 was de
tussentaal van GHC namelijk vervangen door \systemfc, een uitbreiding van System
F met expliciete bewijzen van gelijkheid tussen types.  De mogelijkheid om
\textit{functional dependencies} te implementeren door ze te vertalen in deze
meer expressieve calculus, is tot voor kort niet volledig onderzocht. In 2017
presenteerden Karachalias en Schrijvers het \textit{type inference} algoritme
voor \textit{functional dependencies} en een manier om ze te vertalen naar
\systemfc.

Deze thesis legt een implementatie voor van het bovengenoemde algoritme en geeft
een gedetailleerd overzicht van belangrijke implementatiespecifieke details en
complicaties die zich hebben voorgedaan tijdens de implementatie van
\textit{functional dependencies}.
%
Het doel van deze implementatie is tweevoudig.
%
Ten eerste dient het als een validatie en evaluatie van de specificatie van
Karachalias en Schrijvers. De implementatie heeft verschillende
onnauwkeurigheden in die specificatie onthuld en laten we
zien hoe we aangepakt hebben.
%
Ten tweede dient het als een haalbaarheidsstudie, gericht op het kwantificeren
van de uitdagingen die zich kunnen voordoen bij het integreren van
\textit{functional dependencies} in het huidig Haskell ecosysteem. We tonen aan
dat de benodigde veranderingen minimaal zijn door het type inferentieraamwerk
dat momenteel in GHC aanwezig is te hergebruiken.
