\chapter{Constraint Entailment}
\label{cha:entailment}
%TODO rule names in spec
In this chapter we describe a concrete constraint solver based on the
\textit{OutsideIn(X)}~\cite{outsideinx-modular-type-inference-with-local-assumptions}
inference framework. It is based on the particular constraint solver for the
constraint domain $X$ instantiated with
\[
X = \textit{type classes} + \textit{GADTs} + \textit{type families}
\]
presented in the same work. This solver will be referred to simply as
\textit{OutsideIn(X)} from now on. Because our work does not consider $GADTs$ or other
features that might generate local constraints, the part of the solver that
deals with those constraints is not implemented.

\textit{OutsideIn(X)} was designed with generating evidence in mind but any
concrete specification as to how has been omitted. This chapter contains a
description of the implementation including how evidence is generated during
constraint solving.

\section{Motivation}
\label{sec:motivation}
\textit{"Solving equality contraints is
tricky."}~\cite{outsideinx-modular-type-inference-with-local-assumptions}
Consider the following example:
\begin{verbatim}
    class C a b | a -> b
    instance C Int Bool
\end{verbatim}
$\quad\quad\quad\rightsquigarrow$ $F(Int)$ $\sim$ $Bool$
\begin{verbatim}
    f :: C Int b => b -> Bool
    f x = x
\end{verbatim}

In the end the $b \sim Bool$ wanted constraint somehow has to be solved. This
can be achieved by using the information provided by the class constraint in the
type signature. It gives rise to the given equality constraint $F(Int) \sim
b$. This information can be used to replace $b$ in the wanted constraint
resulting in $F(Int) \sim Bool$. Which is the same as the equality axiom
generated by the instance declaration and it therefore can be easily resolved.
Note that we replace $b$ with $F \; Int$. The wanted constraint that contains no
type families is replaced with one that does. To see why this
could be potentially dangerous, consider the following type signature.
\begin{verbatim}
    g :: C a a => a -> a
\end{verbatim}
This would give rise to the given constraint $F(a) \sim a$. This could be used
to replace any occurrence of $a$ with $F(a)$, but because the latter contains
$a$ itself this can be repeated indefinitely. Being able to replace simple
constraints by constraints containing type families is essential, but is a serious threat to the
termination of the solving algorithm.

The issues described in the last example have been addressed by Schrijvers et
al.~\cite{type-checking-with-open-type-functions} in their work about open type
functions. This solution takes local given constraints produced by GADTs into
account which is a similar problem. The $OutsideIn(X)$ constraint solver
implements a reworked version that is simpler and has worked out
interoperability with type classes.

\section{Overview}

\begin{figure}
\[ % george doesn't like '$$'
P ; \overline{a}_{utch} \vDash \mathcal{C} \rightsquigarrow
(\mathcal{P}_{residual}, \theta, \eta)_\bot
\]
\caption{Constraint Entailment}
\end{figure}

The constraint solver takes the program theory $P$ containing the top-level
information and \textit{given} constraints, a set of untouchable (rigid) type variables
$\overline{a}_{utch}$, (that is, variables the constraint solver is not allowed
to unify), and a set of \textit{wanted} constraints $\mathcal{C}$ the constraint solver is expected to resolve.

Notice that the original version of \textit{OutsideIn(X)} uses \textit{touchable} variables instead, a set of
type variables the constraint solver \textit{is} allowed to unify. We have opted to do
the opposite; this is simply a matter of taste and has no interesting
consequences.

Entailment provides us with residual annotated class constraints that could not
be solved. Residual equality constraints are not allowed: when some cannot be
resolved our solver simply fails. These residual constraints can be quantified over
in the inferred type. Because equality constraints are not allowed in the
source language we disallow residual equality constraints for simplicity.

There are several reasons why entailment could fail which is why the result is
annotated with $\bot$. This usually indicates a fatal failure, meaning that the
compiler has to give up and present an error to the user.

Most importantly, the result of the entailment relation comes in the form of a
type substitution and an evidence solution to be applied to the elaborated
\systemfc ~term to replace the placeholder variables with the inferred types,
coercions, and dictionary terms.

The strategy the solver employs involves computing the fixed point of a set of
rewrite rules applied to the set of given and wanted constraints. In other
words, we repeatedly apply several rewrite rules until no rule applies and the
constraints have reached a normal form where the constraints closely resemble a
type substitution.

Most of these rules have a wanted and given variant. Both often share many
similarities but mainly differ in how they handle evidence. Wanted constraints
are annotated with evidence variables that eventually need to be substituted
with evidence for why these original constraints hold. Every time a rule
rewrites a constraint, a part of the substitution is constructed that mimics the
change in the constraint. One could say the evidence is a trace of all steps the
solver took to prove the constraint. E.g. $c : Maybe \; a \sim Maybe \; b$ could
be simplified to $c': a \sim b$, which would generate $[c \mapsto \langle Maybe
\rangle \; c']$ as evidence for the transition.

With given constraints, essentially the reverse is true. These constraints are
already annotated with the evidence for why they hold. When a constraint is
changed, the evidence will be adapted to hold for the new constraints. E.g.
$\gamma : Maybe \; a \sim Maybe \; b$ will be changed to $(\text{right}\; \gamma) :
a \sim b$.

\section{Rewrite Rules}
In this section we describe and specify the various constraint rewrite rules
used by the constraint solver. If one such rule matches a constraint, this
constraint will be removed from the constraint set and replaced by the output
constraints of the rule. Other than constraints, these rules might produce
additional by-products as a side-effect of matching a constraint. For example,
matching wanted constraints usually produces an evidence substitution.

The syntax we use for the rewrite rules is similar to Haskell guard syntax, we
match on the shape of the constraint, the statement behind the guard,
``$\mid$'', are conditions that need to be checked before we match. $t$ can be
$g$ or $w$, indicating the rule deals with wanted or given constraints
respectively.
\[
\begin{array}{l c l}
rule_t(input)
\\\guard conditions &=& output
\end{array}
\]

We denote single annotated constraints, class constraints and equality
constraints, as part of the input and output with $\mathcal{Q}$ and the given
variety with $\mathcal{Q}_g$.

\subsection{Substituting with Evidence}
\begin{figure}
\fbox{$[a \mapsto \tau](\gamma,\tau') \rightsquigarrow (\gamma',\tau'')$}
\begin{mathpar}
\inferrule*[right=CoSubSame]
{
    ~
}
{
    [a \mapsto \tau](\gamma, a) \rightsquigarrow (\gamma,\tau)
}
\quad
\inferrule*[right=CoSubDiff]
{
    ~
}
{
    [a \mapsto \tau](\gamma, b) \rightsquigarrow (\langle b \rangle, b)
}
\\
\inferrule*[right=CoSubApp]
{
    [a \mapsto \tau](\gamma, \tau_1) \rightsquigarrow (\gamma_1', \tau_1')
    \\
    [a \mapsto \tau](\gamma, \tau_2) \rightsquigarrow (\gamma_2', \tau_2')
}
{
    [a \mapsto \tau](\gamma, \tau_1 \; \tau_2) \rightsquigarrow (\gamma_1' \;
    \gamma_2', \tau_1' \; \tau_2')
}
\\
\inferrule*[right=CoSubTyCon]
{
    ~
}
{
    [a \mapsto \tau](\gamma, T) \rightsquigarrow (\langle T \rangle, T)
}
\\
\inferrule*[right=CoSubTyFam]
{
    [a \mapsto \tau](\gamma, \tau_i) \rightsquigarrow (\gamma_i, \tau_i')
}
{
    [a \mapsto \tau](\gamma, F(\overline{\tau})) \rightsquigarrow
    (F(\overline{\gamma}), F(\overline{\tau}'))
}
\\
\inferrule*[right=CoSubCls]
{
    [a \mapsto \tau](\gamma, \tau_i) \rightsquigarrow (\gamma_i, \tau_i')
}
{
    [a \mapsto \tau](\gamma, TC \; \overline{\tau}) \rightsquigarrow
    (\langle TC \rangle \; \overline{\gamma}, TC \; \overline{\tau}')
}
\end{mathpar}
\caption{Substituting with Evidence}
\label{fig:sub-evidence}
\end{figure}

In many of the rewrite rules that are discussed in this chapter, we perform some
form of substitution on the types in the constraints. Given a coercion that
proves the equality of the type variable and the type that it will be
substituted with, it produces additionally a coercion that proves the equality
of the result type before and after the substitution.

This procedure is specified in Figure~\ref{fig:sub-evidence} and takes a type
variable $a$, a type $\tau$ to substitute it with, and a coercion $\gamma$ that
proves the equality between $a$ and $\tau$. The result is the substituted type,
and a coercion that proves the equality of the type before and after the
substitution. Consider the following example:
\[
    [a \mapsto \tau](\gamma, T \; a) \rightsquigarrow (\langle T \rangle \;
    \gamma, T \; \tau)
\]
If the coercion $\gamma$ proves $a \sim \tau$, then the coercion $\langle T
\rangle \; \gamma$ proves $T \; a \sim T \; \tau$.

\newpage %FIXME
\subsection{Canonicalization}
\label{sec:canonicalization}

Canonicalization, specified in figures~\ref{fig:canon_w} and \ref{fig:canon_g},
transforms a single constraint to a simpler form. Constraints that cannot be
simplified further by canonicalization have a very particular shape and are
called \textit{canonical constraints}. These are specified in
figure~\ref{fig:canon-cs}.

\begin{figure}
\fbox{$\vdash_{can} Q$}
\begin{mathpar}
\inferrule*[right=CEQ]
{
    a \prec u
    \\
    a \notin fv(u)
}
{
    \vdash_{can} a \sim u
}
\quad
\inferrule*[right=CFEQ]
{
    ~
}
{
    \vdash_{can} F(\overline{u}) \sim u
}
\quad
\inferrule*[right=CDICT]
{
    ~
}
{
    \vdash_{can} TC \; \overline{u}
}
\end{mathpar}
\caption{Canonical Constraints}
\label{fig:canon-cs}
\end{figure}

\subsubsection{Decomposing types}

One of the responsibilities of canonicalization is decomposing type
applications. For example $Maybe \; a \sim Maybe \; Bool$ will be split up in
$Maybe \sim Maybe$ and $a \sim Bool$. The first can be trivially resolved because
both types of the constraint are the same. However, if it encountered a
constraint like $Int \sim Bool$ that consists of two different type
constructors, it will be rejected. This is a classic example of a type error
and indicates that the programmer has used an expression of the wrong type. We
will simply fail and present an error to the user.

\subsubsection{Occurrence check}
Constraints like $a \sim [a]$ would, if left to their devices, result in
infinite types and in loss of termination. These issues are detected
during canonicalization and would cause a fatal error and return $\bot$. Note that
this rule $(a \sim u)$ where $a \in fv(u) \;$ matches only if the other type
does not contain any type families.
\begin{verbatim}
    class Coll c e | c -> e
    f :: Coll [a] a => ..
\end{verbatim}
The type signature is this example gives rise to the $F([a]) \sim a$ given
constraint. This is perfectly fine, these situations do not necessarily cause
termination issues under type families.

\subsubsection{Orientation}
Canonicalization uses the $\prec$ relation of figure~\ref{fig:orientation} to put the
constraints in a specific orientation. Type families are put on the left of the
constraint equality. With no families, type variables are put on the left. This
is done so we can concisely control in the other rules what happens to constraints with and without type
families. In the case of constraints with two type
variables a total ordering is used to ensure termination. The exact ordering is not of importance;
its existence is what matters.
When one variable is untouchable and the other is not, the
touchable is preffered on the left simply to make the constraint look more like
a substitution. When they are both untouchable or
touchable, lexicographical order is used.
%TODO actually, unique's are used for order on names, it's faster and cleaner.

One needs to be careful when implementing the ordering so that it does not hold
in both directions, thus risking non-termination.

\begin{figure}
\fbox{$\tau_1 \prec_{a_{utch}} \tau_2$}
\[
\begin{array}{l c l l}
F(\overline{\tau}) &\prec_{a_{utch}}& \tau &\text{when} \; \tau \not\equiv
F'(\overline{\tau}')
\\
a &\prec_{a_{utch}}& b &\text{when} \; a \notin \overline{a}_{utch}, b \in
\overline{a}_{utch}
\\
a &\prec_{a_{utch}}& b &\text{when} \; a < b \; \text{lexicographically}
\\
a &\prec_{a_{utch}}& \tau
\end{array}
\]
\caption{Orientation}
\label{fig:orientation}
\end{figure}

\subsubsection{Flattening}
Type families are not allowed to appear nested as part of other types in
canonical constraints. The flattening procedure extracts those type families and
splits up the constraint. For example: %TODO cite example source
$$
a \sim [F(a)]
$$
becomes
$$
a \sim [\beta], F(a) \sim \beta
$$
where $\beta$ is a fresh type variable. We replace the type family with the
fresh variable $\beta$ and create evidence so we can reconstruct the original
type when we are done solving constraints. In conjunction with the orientation,
this is done to easily control interaction with type families even when they
appear nested as part of other types. It is also essential for termination. If we
considered $a \sim [F(a)]$ a left-to-right substitution, it could be applied an
indefinite amount of times as mentioned in Section~\ref{sec:motivation}.
Replacing $F(a)$ with $\beta$ ensures this can happen only once.

%TODO explain flattening substitution
Canonicalization for given constraints is somewhat different compared to the
other rules for given constraints. It produces a set of additional untouchable
variables, a type substitution and an evidence substitution. The substitutions
are called the \textit{flattening substitutions}. Denoted by $\theta_{flat}$ and
$\eta_{flat}$ for flattening type and evidence substitution respectively. Their
purpose is to undo the flattening at the end of constraint entailment. For the
example above, $[\beta \mapsto F(a)]$ would be generated. This is not needed for
wanted constraints because this happens naturally when the type substitution
resulting from entailment is constructed

The syntax related to the extraction is specified in
figure~\ref{fig:flattening-contexts}. The flattening contexts are simply type
patterns, type family applications or class constraints with a hole (denoted by
$\mathbb{T}$, $\mathbb{F}$, and $\mathbb{C}$ respectively). In practice,
matching on these contexts involves doing depth-first search on the monotype to
find the deepest nested type family and replacing them with the hole. Matching
on $\mathbb{F}[F(\overline{u})]$ means performing this search, matching the rule
if it succeeds and returning $\mathbb{F}$ and $F(\overline{u})$ as an output.
These contexts are implemented simply using functions, which is the easiest way
to enforce that tree-like structures have only one hole.

\begin{figure}
\[
\begin{array}{l c l}
\mathbb{T} &::=& T \mid \mathbb{T} \; \mathbb{T} \mid a \mid \square\\
\mathbb{F} &::=& F \; \overline{\mathbb{T}}\\
\mathbb{C} &::=& TC \; \overline{\mathbb{T}}
\end{array}
\]
\caption{Flattening Contexts}
\label{fig:flattening-contexts}
\end{figure}

\begin{figure}
\small
\fbox{$canon_w(\overline{a}_{utch}, \mathcal{Q}) = {\mathcal{Q}' \mid \eta}_\bot$}
\[
\begin{array}{l c l}
canon_w(\overline{a}_{utch}, c : \tau \sim \tau) &=& \bullet \mid c \mapsto \langle \tau \rangle
\\
canon_w(\overline{a}_{utch}, c : \tau_1 \; \tau_2 \sim \tau_3 \; \tau_4) &=& \{c_1 : \tau_1 \sim
\tau_3, c_2 : \tau_2  \sim \tau_4\} \mid c \mapsto c_1 \; c_2
\\ \where c_1, c_2 \; \text{fresh}
\\
canon_w(\overline{a}_{utch}, c : T_1 \sim T_2)
\\ \guard T_1 \neq T_2 &=& \bot
\\
canon_w(\overline{a}_{utch}, c : a \sim u)
\\ \guard a \in fv(u) \; \&\& \; u \neq a &=& \bot
\\
canon_w(\overline{a}_{utch}, c : \tau_1 \sim \tau_2)
\\ \qquad \mid \tau_2 < \tau_1 &=& c' : \tau_2 \sim \tau_1 \mid c \mapsto \text{sym}
\; c'
\\ \where c' \; \text{fresh}
\\
canon_w(\overline{a}_{utch}, d : \mathbb{D}[F(\overline{u})]) &=& \{ c_1 :
F(\overline{u}) \sim \beta, d_2 : \mathbb{D}[\beta]\} \mid d \mapsto d_2
\triangleright \gamma
\\ \multicolumn{3}{l}{\where \beta, c_1, d_2 \; \text{fresh}, [\beta \mapsto
F(\overline{u})](\text{sym} \; c_1, \mathbb{D}[\beta]) \rightsquigarrow (\gamma,
\pi)}
\\
canon_w(\overline{a}_{utch}, c : \mathbb{F}[F(\overline{u})] \sim \tau) &=&
\{c_1 : F(\overline{u}) \sim \beta, c_2 : \mathbb[\beta] \sim \tau \} \mid c
\mapsto (\text{sym} \; \gamma) \fctrans c_2
\\ \multicolumn{3}{l}{\where \beta, c_1, d_2 \; \text{fresh},
[\beta \mapsto F(\overline{u})](\text{sym} \;
c_1, \mathbb{F}[\beta]) \rightsquigarrow (\gamma, \tau)}
\\
canon_w(\overline{a}_{utch}, c : \tau \sim \mathbb{T}[F(\overline{u})])
\\ \qquad \mid \tau \equiv F'(\overline{u}') \; || \; \tau \equiv \alpha &=&
\{c_1 : F(\overline{u}) \sim \beta, c_2 : \tau \sim \mathbb{T}[\beta]\} \mid c
\mapsto c_2 \fctrans \gamma
\\ \multicolumn{3}{l}{\where \beta, c_1, d_2 \; \text{fresh},
[\beta \mapsto F(\overline{u})](\text{sym} \;
c_1, \mathbb{T}[\beta]) \rightsquigarrow (\gamma, \tau')}
\end{array}
\]
\caption{Canonicalize Wanted Constraints}
\label{fig:canon_w}
\end{figure}

\begin{figure}
\small
\fbox{$canon_g(\overline{a}_{utch}, \mathcal{Q}_g) = {\mathcal{Q}_g' \mid
\overline{a}_{utch} ; \theta_{flat} ; \eta_{flat}}_\bot$}
\[
\begin{array}{l c l}
canon_g(\overline{a}_{utch}, \gamma : \tau \sim \tau) &=& \bullet \mid \bullet ; \bullet ; \bullet
\\
canon_g(\overline{a}_{utch}, \gamma : \tau_1 \; \tau_2 \sim \tau_3 \; \tau_4) &=& \{ \text{left} \;
\gamma : \tau_1 \sim \tau_3, \text{right} \; \gamma : \tau_2 \sim \tau_4\} \mid
\bullet ; \bullet ; \bullet
\\
canon_g(\overline{a}_{utch}, \gamma : T_1 \sim T_2)
\\ \qquad \mid T_1 \neq T_2 &=& \bot
\\
canon_g(\overline{a}_{utch}, \gamma : a \sim u)
\\
\qquad \mid a \in fv(u) \; \&\& \; u \neq a &=& \bot
\\
canon_g(\overline{a}_{utch}, \gamma : \tau_1 \sim \tau_2)
\\ \guard \tau_2 \prec_{\overline{a}_{utch}} \tau_1 &=& \text{sym} \; \gamma :
\tau_2 \sim \tau_1 \mid \bullet ; \bullet ; \bullet
\\
canon_g(\overline{a}_{utch}, t : \mathbb{D}[F(\overline{u})]) &=& \{ d : \mathbb{D}[\beta], c :
F(\overline{u}) \sim \beta\}
\\ \where \beta, c, d \; \text{fresh} && \mid \{\beta\} ; \beta \mapsto F(\overline{u}) ; [c
\mapsto \langle F(\overline{u}) \rangle, d \mapsto t]
\\
canon_g(\overline{a}_{utch}, \gamma : \mathbb{F}[F(\overline{u})] \sim \tau) &=& \{ c_1 :
\mathbb{F}[\beta] \sim \tau, c_2 : F(\overline{u}) \sim \beta\}
\\ \where \beta, c_1, c_2 \; \text{fresh} && \mid \{ \beta \} ; \beta \mapsto F(\overline{u}) ; \; [c_1 \mapsto \gamma,
c_2 \mapsto \langle F(\overline{u}  \rangle)]
\\
canon_g(\overline{a}_{utch}, \gamma : \tau \sim \mathbb{T}[F(\overline{u})])
\\ \guard \tau \equiv F'(\overline{u}') \; || \; \tau \equiv \alpha &=&
\{ c_1 : \tau \sim \mathbb{T}[\beta], c_2 : F(\overline{u}) \sim \beta\}
\\ \where \beta, c_1, c_2 \; \text{fresh} && \mid \{\beta\} ; \beta \mapsto F(\overline{u}) ; [c_1 \mapsto \gamma, c_2
\mapsto \langle F(\overline{u}) \rangle]
\end{array}
\]
\caption{Canonicalize Given Constraints}
\label{fig:canon_g}
\end{figure}

\subsection{Binary interaction}
\label{sec:binary-interaction}
The binary interaction rules are specified in figures~\ref{fig:interact_w} and
\ref{fig:interact_g} and transform two given or wanted \textit{canonical
constraints} to either given or wanted constraints respectively.

For equality constraints, this rule expresses a similar notion as type
substitution. The first constraint is used as a left-to-right substitution and
all occurrences of the left-hand, usually a type variable, are replaced by the
right-hand side in the second constraint. Note that because we only use
canonical constraints, type family applications will only be replaced with type
patterns and we never introduce additional type families. The total number can
only decrease and this is essential for termination.
%TODO mention total order termination stuff?

Binary interactions also remove duplicates of type class constraints. This
makes a lot of sense for wanted constraints: if we solved one, we can also solve
the other and the evidence of $[d_2 \rightarrow d_1$] clearly reflects this
notion. For given constraints however, we do the same but simply throw away the
evidence for one of them. This has no consequences because the coherence
property of type classes is enforced. This means that there can only be one
instance for any given type. The property implies that if the evidence can be
derived in multiple ways, it should semantically be the same. So there is no
loss of information.
%TODO example?

Every rule returns the first constraint it consumes unmodified. This
could be implemented by only consuming the second constraint instead, if desired.

\begin{figure}
\small
\fbox{$interact_w(\mathcal{Q}_1, \mathcal{Q}_2) = \mathcal{Q}' \mid \eta$}
\[
\begin{array}{l c l}
interact_w(c_1 : a \sim u_1, c_2 : a \sim u_2) &=& \{c_1 : a \sim u_1, c_2' :
u_1 \sim u_2\} \mid c_2 \mapsto c_1 \fctrans c_2'
\\ \where c_2' \; \text{fresh}
\\
interact_w(c_1 : a \sim u_1, c_2 : b \sim u_2)
\\ \guard a \in fv(u_2 )&=& \{c_1 : a  \sim u_1, c_2' : b \sim \tau\} \mid
c_2 \mapsto c_2' \fctrans \text{sym} \; \gamma
\\ \where [a \mapsto u_1](c_1, u_2) \rightsquigarrow (\gamma, \tau),
\\ \whereindent c_2' \; \text{fresh}
\\
interact_w(c_1 : a \sim u_1, c_2 : F(\overline{u}) \sim u_2)
\\ \guard a \in fv(\overline{u}, u_2) &=& \{c_1 : a \sim u_1, c_2' : \tau_1
\sim \tau_2\} \mid c_2 \mapsto \gamma_1 \fctrans c_2' \fctrans \text{sym} \;
\gamma_2
\\ \where [a \mapsto u_1](c_1, F(\overline{u})) \rightsquigarrow
(\gamma_1, \tau_1),
\\ \whereindent [a \mapsto u_1](c_1, u_2) \rightsquigarrow
(\gamma_2, \tau_2),
\\ \whereindent c_2' \; \text{fresh}
\\
interact_w(c : a \sim u, d : TC \; \overline{u})
\\ \guard a \in fv(\overline{u}) &=& \{c : a \sim u, d' : \pi\} \mid d
\mapsto d' \triangleright \text{sym} \; \gamma
\\ \where [a \mapsto u](c, TC \; \overline{u}) \rightsquigarrow
(\gamma, \pi)
\\ \whereindent d' \; \text{fresh}
\\
interact_w(c_1 : F(\overline{u}) \sim u_1, c_2 : F(\overline{u}) \sim u_2) &=&
\{c_1 : F(\overline{u}) \sim u_1, c_2' : u_1 \sim u_2\} \mid c_2 \mapsto c_1
\fctrans c_2'
\\
interact_w(d_1 : TC \; \overline{u}, d_2 : TC \; \overline{u}) &=& d_1 : TC \;
\overline{u} \mid d_2 \mapsto d_1
\end{array}
\]
\caption{Wanted Binary Interaction Rules}
\label{fig:interact_w}
\end{figure}

\begin{figure}
\small
\fbox{$interact_g(\mathcal{Q}_1, \mathcal{Q}_2) = \mathcal{Q}'$}
\[
\begin{array}{l c l}
interact_g(\gamma_1 : a \sim u_1, \gamma_2 : a \sim u_2) &=& \{\gamma_1 : a \sim
u_1, \text{sym} \; \gamma_1 \fctrans \gamma_2 : u_1 \sim u_2\}
\\
interact_g(\gamma_1 : a \sim u_1, \gamma_2 : b \sim u_2)
\\ \guard a \in fv(u_2 )&=& \{\gamma_1 : a  \sim u_1, \gamma_2 \fctrans \gamma :
b \sim \tau\}
\\ \where [a \mapsto u_1](\gamma_1, u_2) \rightsquigarrow (\gamma, \tau)
\\
interact_g(\gamma_1 : a \sim u_1, \gamma_2 : F(\overline{u}) \sim u_2)
\\ \guard a \in fv(\overline{u}, u_2) &=& \{\gamma_1 : a \sim u_1, \text{sym} \;
\gamma_1' \fctrans \gamma_2 \fctrans \gamma_2': \tau_1 \sim \tau_2\}
\\ \where [a \mapsto u_1](\gamma_1, F(\overline{u})) \rightsquigarrow
(\gamma_1', \tau_1)
\\ \whereindent [a \mapsto u_1](\gamma_1, u_2) \rightsquigarrow
(\gamma_2', \tau_2)
\\
interact_g(\gamma : a \sim u, t : TC \; \overline{u})
\\ \qquad \mid a \in fv(\overline{u}) &=& \{\gamma : a \sim u, t \triangleright
\gamma' : \pi\}
\\ \where [a \mapsto u](\gamma, TC \; \overline{u}) \rightsquigarrow
(\gamma', \pi)
\\
%TODO guard for equality?
interact_g(\gamma_1 : F(\overline{u}) \sim u_1, \gamma_2 : F(\overline{u}) \sim
u_2) &=& \{\gamma_1 : F(\overline{u}) \sim u_1, \text{sym} \; \gamma_1 \fctrans
\gamma_2: u_1 \sim u_2\}
\\
interact_g(t_1 : TC \; \overline{u}, t_2 : TC \; \overline{u}) &=& \{t_1 : TC \;
\overline{u}\}
\end{array}
\]
\caption{Given Binary Interaction Rules}
\label{fig:interact_g}
\end{figure}

\subsection{Simplification}
Simplification is highly similar to binary interaction but it transforms a
single wanted constraint using a given constraint, producing a new wanted
constraint. This rule does not consume the given constraint, only the wanted.

At first glance it might seem that all of the binary interaction rules are here
as well. But the \textit{simplifies} rules are unidirectional and rules like
\[
\begin{array}{l}
simplifies(\gamma_1 : F(\overline{u}) \sim u_2, c_2 : a \sim u_1)
\\
simplifies(t : TC \; \overline{u}, c : a \sim u_1)
\end{array}
\]
are actually missing. These are not allowed because they introduce additional
type families and type class constraints that need to be solved. This is
dangerous for termination and does not contribute additional information that we
cannot derive from $a \sim u_1$.

\begin{figure}
\small
\fbox{$simplifies(\mathcal{Q}_g, \mathcal{Q}) = \mathcal{Q}' \mid \eta$}
\[
\begin{array}{l c l}
simplifies(\gamma : a \sim u_1, c : a \sim u_2) &=& \{c' : u_1 \sim u_2\} \mid c
\mapsto \gamma \fctrans c'
\\ \where c' \; \text{fresh}
\\
simplifies(\gamma_1 : a \sim u_1, c : b \sim u_2)
\\ \guard a \in fv(u_2) &=& \{c' : b \sim \tau\} \mid c \mapsto c' \fctrans
\text{sym} \; \gamma_2
\\ \where [a \sim u_1](\gamma_1, u_2) \rightsquigarrow (\gamma_2, \tau)
\\ \whereindent c' \; \text{fresh}
\\
simplifies(\gamma_1 : a \sim u_1, c_2 : F(\overline{u}) \sim u_2)
\\ \guard a \in fv(\overline{u}) &=& \{c': \tau \sim u_2\} \mid c_2 \mapsto
\gamma_2 \fctrans c'
\\ \where [a \mapsto u_1](\gamma_1, F(\overline{u})) \rightsquigarrow (\gamma_2,
\tau)
\\ \whereindent c' \; \text{fresh}
\\
simplifies(\gamma_1 : a \sim u_1, d : TC \; \overline{u})
\\ \guard a \in fv(\overline{u}) &=& \{d' : \pi\} \mid d \mapsto
d' \triangleright \text{sym} \; \gamma_2
\\ \where [a \mapsto u_1](\gamma_1, TC \; \overline{u}) \rightsquigarrow
(\gamma_2, \pi)
\\ \whereindent d' \; \text{fresh}
\\
simplifies(\gamma_1 : F(\overline{u}) \sim u_1, c : F(\overline{u}) \sim u_2)
&=& \{c' : u_1 \sim u_2\} \mid c \mapsto \gamma_1 \fctrans c'
\\ \where c' \; \text{fresh}
\\
simplifies(t : TC \; \overline{u}, d : TC \; \overline{u}) &=& \bullet \mid d
\mapsto t
\end{array}
\]
\end{figure}

\subsection{Top-level Reactions}
The last set of rules we present are the top-level reaction rules, depicted
in figure~\ref{fig:topreact}. These rules are used to interact with top-level
class instance schemes and type equality axioms that get introduced by instance
declarations.

%TODO citation backward chaining?
For solving class constraints the standard \textit{backwards-chaining} method is
used. This method is specified in rule $topreact_w$. Instance declarations
are interpreted as logical implications, $\texttt{instance} \; \overline{\pi}
\Rightarrow \pi$ means that if all of the constraints in the premise
$\overline{\pi}$ are known or can be proven, we can prove $\pi$. Because we
start from the goal, we look for an declaration that proves it, assume that it
holds and attempt to prove the premises instead. Effectively working our way
backwards through the implications, hence the name \textit{backwards-chaining}.
For example:
\begin{verbatim}
    instance Eq Int
    instance Eq Bool
    instance (Eq a, Eq b) => Eq (a, b)
\end{verbatim}
To prove the wanted constraint \textit{Eq (Int, Bool)} the instance declaration
for \textit{Eq (a, b)} is used with $a$ instantiated with $Int$ and $b$ with
$Bool$. As a consequence the constraints in the premise are added as new wanted
constraints. Both of which can be solved in the next iterations where they will
match the instance declarations for \textit{Eq Int} and \textit{Eq Bool}
respectively. As these instances do not have premises, they will be completely
resolved without giving rise to new wanted constraints.

%TODO cite 2007a, mention solving in both directions
Equality axioms are are handled in a very similar way as binary interaction or
simplification. These are simply used as left-to-right substitutions. Unlike
wanted or given constraints, equality axioms and constraint schemes are
polymorphic. Simple Hindley-Milner\cite{hindley}\cite{damas-milner} unification,
as depicted in
figure~\ref{fig:unify}, is used to produce a substitution that maps the
abstracted over type variables to the instantiated types. This why the free type
variables from the wanted or given constraints are included as untouchables. So
the substitution does not end up mapping type variables in the constraint to
those from the program theory because these may never end up in the resulting
\systemfc ~code. The axiom or constraint scheme does not match if
unification fails.

The top-level reaction rules are the only rules allowed to introduce additional
type families and type class constraints. To make sure that they do not harm
termination several restrictions are imposed on class and instance declarations.
More on these conditions is discussed in chapter~\ref{cha:conditions}.

\begin{figure}
\fbox{$topreact_w(\overline{a}_{utch}, P, \mathcal{Q}) = \mathcal{Q}' \mid \eta$}
\begin{mathpar}
\inferrule*[right=TopreactClsW]
{
    (d_I : \forall \overline{a} \overline{b}. \; \overline{\pi} \Rightarrow TC
    \; \overline{u}') \in P
    \\
    \overline{d}, \overline{b}' \; \text{fresh}
    \\
    \overline{a}_{utch}' = \overline{a}_{utch}, fv(\overline{u})
    \\
    \theta = [b \mapsto b'] \cdot unify(\overline{a}_{utch}', \overline{u \sim
    u'})
}
{
    topreact_w(\overline{a}_{utch}, P, d : TC \; \overline{u}) = \overline{d
    : \theta(\pi)} \mid d \mapsto d_I \; \theta(\overline{a} \overline{b}) \;
    \overline{d}
}
\end{mathpar}
\fbox{$topreact_w(\overline{a}_{utch}, P, \mathcal{Q}) = \mathcal{Q}' \mid \eta$}
\begin{mathpar}
\inferrule*[right=TopreactEqW]
{
    (g \; \overline{a} : F(\overline{u}') \sim \tau) \in P
    \\
    \overline{a}_{utch}' = \overline{a}_{utch}, fv(\overline{u},u)
    \\
    \theta = unify(\overline{a}_{utch}', \overline{u \sim u'})
}
{
    topreact_w(\overline{a}_{utch}, P, c : F(\overline{u}) \sim u) = c' :
    \theta(\tau) \sim u \mid c \mapsto g \; \theta(\overline{a}) \fctrans c'
}
\end{mathpar}
\fbox{$topreact_g(\overline{a}_{utch}, P,\mathcal{Q}_g) = \mathcal{Q}_g'$}
\begin{mathpar}
\inferrule*[right=TopreactEqG]
{
    (g \; \overline{a} : F(\overline{u}') \sim \tau) \in P
    \\
    \overline{a}_{utch}' = \overline{a}_{utch}, fv(\overline{u},u)
    \\
    \theta = unify(\overline{a}_{utch}', \overline{u \sim u'})
}
{
    topreact_g(\overline{a}_{utch}, P, \gamma : F(\overline{u}) \sim u) =
    \text{sym} \; (g \; \theta(\overline{a})) \fctrans \gamma : \theta(u') \sim
    u
}
\end{mathpar}
\caption{Top-level reaction rules}
\label{fig:topreact}
\end{figure}

\begin{figure}
%TODO appendix?
\fbox{$unify(\overline{a}; \overline{\phi}) = \theta_\bot$}
\[
\begin{array}{l c l}
unify(\overline{a}; \bullet) &=& \bullet
\\
unify(\overline{a}; \overline{\phi}, b \sim b) &=& unify(\overline{a};
\overline{\phi})
\\
unify(\overline{a}; \overline{\phi}, T \sim T) &=& unify(\overline{a};
\overline{\phi})
\\
unify(\overline{a}; \overline{\phi}, b \sim \tau)
\\ \guard b \notin \overline{a}, b \notin fv(\tau) &=& unify(\overline{a};
\theta(\overline{\phi})) \cdot \theta
\\ \where \theta = [b \mapsto \tau]
\\
unify(\overline{a}; \overline{\phi}, \tau \sim b)
\\ \guard b \notin \overline{a}, b \notin fv(\tau) &=& unify(\overline{a};
\theta(\overline{\phi})) \cdot \theta
\\ \where \theta = [b \mapsto \tau]
\\
unify(\overline{a}; \overline{\phi}, \tau_1 \; \tau_2 \sim \tau_3 \; \tau_4) &=&
unify(\overline{a}; \overline{\phi}, \tau_1 \sim \tau_3, \tau_2 \sim \tau_4)
\\
unify(\overline{a}; \overline{\phi}) &=& \bot
\end{array}
\]
\caption{Hindley-Milner Unification}
\label{fig:unify}
\end{figure}

\newpage
\section{Solver structure}

\subsection{Rule order}
Even though the algorithm was designed to be independent of the order in which
the rewrite rules fire, a decision has been made on a specific order. Because it
is strictly required for the soundness of the algorithm or so that the rules for
which we know that they don't apply to anything anymore are no longer attempted.
The latter is not required by any means but it did not make the solver more
complex.

Every rule except \textbf{canonicalization} expects canonical
constraints. Therefore this rule is exhaustively applied first. Afterwards,
other rules might produce additional rules that are not canonical and
canonicalization is applied just to the output of all other rules but not to the
entire set of constraints.

The set of \textbf{given} constraints is in no way affected by rewrite rules
involving wanted constraints and therefore we exhaustively apply all rules
exclusive to given constraints first.

%TODO rename topreact if name doesn't change
Lastly we do the same for all \textbf{wanted} constraints but these can't be
applied in any order. We must refrain from applying $topreact$ to class
constraints until the very end when no other rules apply. This is to prevent
issues related to non-determinism over given class constraints. For example:
\begin{verbatim}
    instance Eq a => Eq [a]
    f :: Eq [a] => [a] -> Bool
    f x = x == x
\end{verbatim}
The usage of \texttt{==} gives rise to the wanted constraint $Eq \; [a]$. This
could react with the instance declaration. This would result in the new wanted
constraint $Eq \; a$ for which we have no way of solving. However $Eq \; [a]$ could
easily be $simplified$ away using the given constraint $Eq \; [a]$ from the type
signature. Therefore, $topreact$ is applied last to prevent such situations.

\newpage
\subsection{Rule combinators}
In this section we define several abstractions for composing the rewrite rules
that make up the solver. For the sake of brevity, we omit the inputs and outputs
of the rewrite rules. For constraints, the inputs are replaced by the outputs.
All other side-effects are accumulated in a global state and returned as part of
the final result (except for the generated untouchables) in addition to the
unsolved wanted constraints.

\begin{itemize}
  \item $rule_1 \| rule_2$: We attempt to apply the first rule. If this rule
  does not match any constraint, we apply the second one instead.
  \item $fix(rule)$: The rule is applied repeatedly until it does not match any
  constraint.
  \item $rule_1 \wedge rule_2$: We attempt to apply the first rule and
  regardless of whether it fails, we attempt to apply the second one after.
\end{itemize}

Using these combinators, we can define the solver as follows:
\begin{figure}
\[
\begin{array}{l}
  solve = fix(canon_w) \wedge fix(canon_g) \wedge fix(givens) \wedge
  fix(wanteds)
  \\ 
  \begin{array}{l l c l}
    \where &givens &=& (interact_g \| topreact_g) \wedge fix(canon_w)
    \\
    &wanteds &=& (interact_w \| simplifies \| topreact_w) \wedge fix(canon_g)
  \end{array}
\end{array}
\]
\caption{The solver}
\end{figure}

\subsection{Building a substitution}

\begin{figure}
\fbox{$P ; \overline{a}_{utch} \vDash \mathcal{C} \rightsquigarrow
(\mathcal{P}_{residual}, \theta, \eta)_\bot$}
\begin{mathpar}
\inferrule*[right=Entail]
{
    solve(\overline{a}_{utch}, P, \mathcal{C}) \rightsquigarrow
    {\mathcal{C}_{residual} \mid \theta_{flat} ; \eta_{flat} ; \eta}_\bot
    \\
    \mathcal{C}_{flat} = \theta_{flat}(C_{residual})
    \\
    \mathcal{E}_{flat}, \mathcal{P}_{flat} \equiv \mathcal{C}_{flat}
    \\
    (\theta, \eta_{refl})_\bot = toSubst(\overline{a}_{utch}; \mathcal{E}_{flat})
}
{
    P ; \overline{a}_{utch} \vDash \mathcal{C} \rightsquigarrow
    (\theta(\mathcal{P}_{flat}), \theta, \eta_{refl} \cdot
    \eta_{flat}(\eta))_\bot
}
\end{mathpar}
\caption{Constraint Entailment}
\label{fig:entail}
\end{figure}
%TODO some sort of occurscheck here as well i think

After we get the normalized set of constraint with respect to the rewrite rules,
i.e. no rule applies to these constraints, all that is left to do is to apply
the flatting substitutions to the constraints and the evidence to undo the
flattening done by canonicalization. Afterwards, the resulting constraints are
split up in equality and type class constraints and these equality constraints
are used to build the type substitution. This final procedure of the entailment
relation is specified in figure~\ref{fig:entail}. The normalized and unflattened
equality constraints, denoted by $\mathcal{E}_{flat}$, should be very similar to
a substitution already. $\{ c_1 : a \sim \tau_1, c_2 : F \; \tau_2 \sim b,
\mathellipsis \}$, where $a$ and $b$ are not untouchable, simply needs to become
$[a \mapsto \tau_1, b \mapsto F \; \tau_2, \mathellipsis]$ and the evidence
simply becomes the reflection of the image of the type substitution $[ c_2
\mapsto \langle \tau_2 \rangle, c_2 \mapsto \langle \tau_2 \rangle,
\mathellipsis]$.

The $toSubst$ function specified in figure~\ref{fig:to-subst} constructs the
type substitution and additional evidence out of the constraints
$\mathcal{E}_{flat}$. The constraints $\mathcal{E}_{flat}$ and the resulting
substitution must have certain properties. Every constraint must have a shape
like $a \sim \tau$ or $\tau \sim a$ where $a \notin \overline{a}_{utch}$ and $a
\notin fv(\tau)$. $\mathcal{E}_{flat}$ should be void of inconsistencies like
$\{a \sim Int, a \sim Bool\}$ and the resulting substitution must be idempotent.
The procedure $toSubst$, inspired by unification, satisfies and verifies these
properties. Invalid constraints simply float upwards as residual equality
constraints and $toSubst$ presents an error to the user if it fails to solve all
constraints.

Note that this procedure could fail with a residual equality constraint like $F
\; Int \sim F \; Bool$, which isn't necessarily inconsistent. In practice, one
could quantify over these constraints but because we disallow equality
constraints in the source syntax, we simply fail as with inconsistent constraints
like $Bool \sim Int$.
\begin{figure}
\[
\begin{array}{l c l}
toSubst(\overline{a}_{utch}; \bullet) &=& (\bullet, \bullet)
\\
toSubst(\overline{a}_{utch}; \mathcal{E}, c : \tau \sim \tau) &=&
(\theta, \eta \cdot [c \mapsto \langle \tau \rangle] )
\\ \where (\theta, \eta) = toSubst(\overline{a}_{utch}, \mathcal{E})
\\
toSubst(\overline{a}_{utch}; \mathcal{E}, c : a \sim \tau)
\\ \guard a \notin \overline{a}_{utch} &=& (\theta \cdot [a \mapsto \tau], \eta
\cdot [c \mapsto \langle \tau \rangle])
\\ \where (\theta, \eta) = toSubst(\overline{a}_{utch}, [a \mapsto
\tau](\mathcal{E}))
\\
toSubst(\overline{a}_{utch}; \mathcal{E}, c : \tau \sim a)
\\ \guard a \notin \overline{a}_{utch} &=& (\theta \cdot [a \mapsto \tau], \eta
\cdot [c \mapsto \langle \tau \rangle])
\\ \where (\theta, \eta) = toSubst(\overline{a}_{utch}, [a \mapsto
\tau](\mathcal{E}))
\\
toSubst(\overline{a}_{utch}; \mathcal{E}) &=& \bot
\end{array}
\]
\caption{Build the substitution}
\label{fig:to-subst}
\end{figure}
