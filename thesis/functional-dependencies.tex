\section{Functional Dependencies}
\label{sec:functional-dependencies}
Multi-parameter type classes are syntactically straightforward. The class
declaration simply specifies additional type parameters that can be used
in the class methods. The intuition behind these type classes is that they
describe relationships between the type parameters.

For example, if we would like to define polymorphic methods for collections and their
elements, we could define a multi-parameter class \texttt{Collection} where
parameter \texttt{c} represents the collection type and \texttt{e} the
element type.
\begin{verbatim}
    class Collection c e where
      singleton :: e -> c
\end{verbatim}
This example only specifies one method \texttt{singleton} which
should put an element in its own collection.
The following example is an instance declaration for lists of a polymorphic type
\texttt{a}.
\begin{verbatim}
    instance Collection [a] a where
      singleton x = [x]
\end{verbatim}
%
Using the methods of these type classes can prove to be quite problematic. The
following example is a function that applies the \texttt{singleton} function
twice to the given argument.

\begin{verbatim}
    singleton2 :: (Collection c2 c1, Collection c1 e) => e => c2
    singleton2 e = singleton (singleton e)
\end{verbatim}
%
Notice that the type signature contains a type
\texttt{c1} that appears in the type class constraints but not in the right-hand
side of the type signature. No matter the context in which this function is
used, it is impossible to uniquely determine type \texttt{c1}.

Even if there is only one available instance that could possibly satisfy these
constraints and provide a type to assume the role of \texttt{c1}, the exact
type could never be determined because type classes are open. The programmer
is free to implement new instance declarations for every type class.

If adding a new instance declaration could cause the type inferred for
\texttt{c1} to change, so would the associated behavior and this would be
unacceptable (the behavior of the program would be determined not by the programmer but by the compiler). This problem is known as {\em incoherence}.
Single-parameter type classes can exhibit this behavior
too but this issue is very prevalent with the
multi-parameter variant. Therefore, a solution has been implemented: functional
dependencies:
\begin{verbatim}
    class Collection c e | c -> e where
      singleton :: e -> c
\end{verbatim}
The \texttt{c -> e} that was added to the \texttt{Collection} class
declaration is the functional dependency between \texttt{c} and \texttt{e}. This
enforces that \texttt{e} has to be uniquely determined by \texttt{c}.
This means that in our \texttt{collection2} example, \texttt{c1} would no longer
be ambiguous because it is uniquely determined by \texttt{c2}.

The next example is a more concrete example of what the functional dependency
enforces. Let \texttt{ByteArray} be a type that contains a sequence of
\texttt{Byte}s or \texttt{Bit}s, depending on how it is accessed.
\begin{verbatim}
    instance Collection ByteArray Byte
        singleton = ...
    instance Collection ByteArray Bit
        singleton = ...
\end{verbatim}
Both of these types of elements can warrant their own instance declaration for
\texttt{Collection}. But if the functional dependency is specified, we have to
be able to determine the element type \texttt{e} when only the collection
type \texttt{c} is known and therefore only one of these instances is allowed
and the other one has to be removed in order to satisfy the type checker.

And exactly because this property is enforced, the \texttt{singleton2} issue was
resolved. In short, the type checker will not allow the programmer to add instance
declarations that would otherwise cause an ambiguity. The type system ensures
there can be only one instance declaration and can therefore use this
information during type inference.

% TODO certains things are still rejected blalba, next section we'll say more
