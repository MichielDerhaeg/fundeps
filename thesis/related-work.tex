% vim:ft=tex
\section{Related Work}

\george{Cite them the first time you mention their name. Take the citations off
the titles, since they show up in the table of contents. I gave an example of this in TAPL}

% TODO where and how to i cite these?
\subsection{Types and Programming Languages}

Types and Programming Languages~\cite{pierce2002types} provides a comprehensive introduction to type system and programming
language theory. It introduces the basic concepts and notational conventions
required to comprehend anything slightly more advanced related to type systems.
It provided an intuitive and formal description of System F and of
Hindley-Milner, both of which are fundamental for understanding Haskell's type
system and it's compilation process.
\subsection{System F with Type Equality
Coercions\cite{Sulzmann:2007:SFT:1190315.1190324}}
This paper originally introduced System $F_C$
\subsection{How to make \textit{ad-hoc} polymorphism less \textit{ad
hoc}\cite{Wadler:1989:MAP:75277.75283}}
historical proposal
\subsection{Type Classes Functional Dependencies\cite{Jones00typeclasses}}
historical proposal
\subsection{Elaboration on Functional
Dependencies\cite{Karachalias:2017:EFD:3156695.3122966}}

\subsection{Quantified Class Constraints\cite{Bottu:2017:QCC:3156695.3122967}}
The relation to this work is artificial. The original prototype compiler my work
is based on was built to support the type system feature that is presented in
this paper. This paper served as a specification that facilitated easier
navigation through the existing codebase as the code closely corresponded to the
presented theory.
\subsection{OutsideIn(X)\cite{outsideinx-modular-type-inference-with-local-assumptions}}
This paper served as an excellent guide to the current state of affairs on how
type inference is done in GHC. And also as an introduction to the general
structure of more advanced type inference algorithms. Lastly, this paper
describes a type inference algorithm that supports both GADTs and Type Families
and will therefore be used as a specification to implement both of these
features in the prototype compiler to study the interaction between these and
our implementation.
