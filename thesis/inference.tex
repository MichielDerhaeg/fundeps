\chapter{Inference and Elaboration}
\label{cha:inference}
Like most systems similar to Hindley-Milner, type inference occurs in two
distinct phases. The first involves constraint generation, and the second
involves solving those constraints. The latter will be discussed in
chapter~\ref{cha:entailment} about constraint entailment.

During constraint generation the Haskell code is simultaneously elaborated into
System $F_C$. However, even though System $F_C$ is an explicitly typed language,
the types it should be annotated with are not yet known. Therefore every
expression will be assigned a "placeholder" unification type variable that might
be substituted with the actual type after the constraints have been solved.

A similar approach is used for type classes. As mentioned previously, type
classes are elaborated into System $F_C$ as dictionary terms that hold the class
methods. Because these are also not yet known, placeholder dictionary variables
are used instead.

%TODO explain fd's after generation, call it determinacy

\section{Source Syntax}
\begin{figure}
\begin{align*}
    a, b, \alpha, \beta &::= \; \langle type \; variable \rangle \\
    x, f                &::= \; \langle term \; variable \rangle \\
    T                   &::= \; \langle type \; constructor \rangle \\
    K                   &::= \; \langle data \; constructor \rangle \\
    TC                  &::= \; \langle class \; constructor \rangle \\
    F                   &::= \; \langle type \; family \rangle \\
    \\
    pgm &::= \overline{decl} &program\\
    decl &::= cls \mid inst \mid data \mid val &declaration\\
    cls &::= \textbf{class} \; \forall \overline{a} \overline{b}. \;
    \overline{\pi} \Rightarrow TC \; \overline{a} \mid \overline{fd}^m
    \textbf{where} \; f :: \sigma &class\\
    inst &::= \textbf{instance} \; \forall \overline{a} \overline{b}. \;
    \overline{\pi} \Rightarrow TC \; \overline{u} \; \textbf{where} \; f = e
    &instance\\
    data &::= \textbf{data} \; T \; \overline{a} = \overline{K \; \overline{a}
    \;} &data\\
    val &::= x = e \mid x :: \sigma = e &value \; binding\\
    fd &::= a_1, \mathellipsis, a_n \rightarrow a_0 &fundep\\
    \\
    e &::= x \mid K \mid e_1 \; e_2 \mid \lambda x. \; e \mid \textbf{let} \; x
    = e_2 \; \textbf{in} \; e_2 \\
    &\quad \mid \textbf{case} \; e_{scr} \; \textbf{of} \; \overline{K \; \overline{x}
    \rightarrow e} &term \\
    \\
    \sigma &::= \rho \mid \forall a. \; \sigma &polytype \\
    \rho &::= \tau \mid \pi \Rightarrow \rho &qualified \; type \\
    \tau &::= a \mid T \mid \tau_1 \; \tau_2 \mid F(\overline{\tau})
         &monotype \\
    u &::= a \mid T \mid u_1 \; u_2 &type \; pattern \\
    \\
    \pi &::= TC \; \overline{\tau} &class \; constraint \\
    \\
\end{align*}
%TODO put scheme and equality constraint with inference syntax?
\caption{Source syntax}
\label{fig:source-syntax}
\end{figure}
The syntax of our Haskell-like source programs is given in
figure~\ref{fig:source-syntax}. It is highly similar to Haskell '98 except for
several additions and simplifications. Type classes can have
multiple type parameters instead of one. In class and instance declarations we
differentiate between $\overline{a}$, the type variables that occur in the type
class parameters, and $\overline{b}$, the type variables that solely occur in the
context $\overline{\pi}$. Reason being that this doesn't have to be an issue if
these $\overline{b}$ can be uniquely determined by functional dependencies.

Class declarations can now be annotated with functional dependencies of the form
$a_1, \mathellipsis, a_n \rightarrow a_0$, where $a_1, \mathellipsis, a_n$ is
commonly referred to the \textit{domain} and $a_0$ as the \textit{range} of the
functional dependency. GHC allows for functional dependencies of the form $a
\rightarrow b \; c$ with multiple types to the right of the arrow. These are
called \textit{multi-range} functional dependencies, but these can be translated
into the \textit{single-range} variant \cite{fundeps-chrs} and are therefore not
considered in this work.

Expressions consist of a simple $\lambda$-calculus extended with ADT data types
and let expressions.

%TODO more about type families?
Monotypes have been extended with type family applications $F(\overline{\tau})$
\cite{associated-types-with-class} similar to System $F_C$. These are not
allowed in the source text and are only used internally. Therefore, we also
define type patterns denoted by $u$. These are simple monotypes that do not
contain any type families. On top of that we have qualified types and polytypes.
Even though GHC allows types to be qualified with equality constraints in
addition to type class constraints, we only allow class constraints for
simplicity. Note that there is no mention of arrow types, these are considered
type constructors. The type $a \rightarrow b$ would in practice look like
$((\rightarrow) \; a) \; b$ with $(\rightarrow)$ a primitive for the arrow type
constructor.

\begin{figure}
\begin{align*}
    P &::= \langle \mathcal{S}, \mathcal{A}, \mathcal{C}_g \rangle &program \;
    theory
    \\
    \\
    \mathcal{C} &::= \bullet \mid \mathcal{C}, d : \pi \mid \mathcal{C}, c
    : \phi &wanted \; constraints
    \\
    \mathcal{P} &::= \bullet \mid \mathcal{P}, d : \pi &wanted \; class \;
    constraints
    \\
    \mathcal{E} &::= \bullet \mid \mathcal{E}, c : \phi &wanted \; equality
    \; constraints
    \\
    \\
    \mathcal{C}_g &::= \bullet \mid \mathcal{C}_g, t : \pi \mid \mathcal{C}_g,
    \gamma: \phi &given \; constraints
    \\
    \\
    \mathcal{A} &::= \bullet \mid \mathcal{A}, g \; \overline{a} :
    F(\overline{u}) \sim \tau &equality \; axioms
    \\
    \mathcal{S} &::= \bullet \mid \mathcal{S}, d : S &annotated \; constraint \;
    schemes
    \\ 
    \\
    \phi &::= \tau_1 \sim \tau_2 &equality \; constraint
    \\
    S &::= \forall \overline{a}. \overline{\pi} \Rightarrow \pi &constraint \;
    scheme
    \\
    \\
    \eta &::= \bullet \mid [d \mapsto t] \cdot \eta \mid [c \mapsto \gamma]
    \cdot \eta &evidence \; substitution
    \\
    \theta &::= \bullet \mid [\alpha \mapsto \tau] \cdot \theta &type \;
    substitution
    \\
    \Gamma &::= \bullet \mid \Gamma, x : \sigma \mid \Gamma, a : k &typing \; environment
    \\
    \\
    d &::= \langle dictionary \; variable \rangle
\end{align*}
\caption{Inference syntax}
\label{fig:inference-syntax}
\end{figure}
\subsection{Inference Syntax}
%TODO maybe put tyfam extended monotypes here?
%TODO tyfam info should be put somewhere too
Figure~\ref{fig:inference-syntax} describes additional syntax used by the type
checker during inference.

Constraint schemes $S$ capture implications generated by instance declarations
and unlike many formalizations not by class declarations. Equality constraints
denoted by $\phi$ are of the form $\tau_1 \sim \tau_2$ which simply means that
type $\tau_1$ should be equal to $\tau_2$. As we will see in
section~\ref{sec:fundeps}, functional dependencies give rise to equality axioms,
denoted by $\mathcal{A}$. These are simply the Haskell counterpart of System
$F_C$ top-level equality axioms and are semantically equivalent, note that $g$
is an $F_C$ axiom variable in either case. Class constraint, equality
constraints and constraint schemes can be annotated with evidence, which are
dictionary variables, coercion variables and again dictionary variables
respectively. Sets of these are denoted with the calligraphic $\mathcal{P}$,
$\mathcal{E}$ and $\mathcal{S}$ respectively. A set of both annotated class and
equality constraints are denoted by $\mathcal{C}$. When these constraints are
given constraints we use subscript $g$ and these can be annotated with full
$F_C$ terms and coercions. This won't make a difference during constraint
generation but has it's use during entailment.

The program theory, denoted by $P$ is a triple of annotated constraint schemes,
equality axioms and a set of local given constraints. It gathers the all of the
constraints schemes and equality axioms from instance declarations and the local
given constraints when under a qualified type.

The typing environment is standard and stores types of term variables and type
variables with their kinds.

\section{Determinacy}
\label{sec:determinacy}
The determinacy relation is one of the key idea's of the formalization of
functional dependencies used in the work of Karachalias and Schrijvers~\cite{Karachalias:2017:EFD:3156695.3122966}.  It takes the form of
$det(\overline{a},\overline{\pi}) = \theta$ and can be interpreted as follows:
It produces a substitution that replaces the type variables of $\overline{\pi}$
with entirely equivalent types that only use type variables from $\overline{a}$.
It does so by leveraging the functional dependencies of the type class
constraints in question. Every functional dependency gives rise to a type family
that maps the domain to the range of the dependency, allowing us to
\textit{determine} the range using the domain, hence the name
\textit{determinacy}. Consider this simple example:

\[
\begin{array}{l c l}
    \texttt{class C a b | a -> b} &\rightsquigarrow& \textbf{type} \; F_C(a) \\
    \texttt{instance C Int Bool} &\rightsquigarrow& \textbf{axiom} \; g :
    F_C(Int) \sim Bool \\
    \\
\end{array}
\]
\[
    det(c, C \; c \; d) = [d \mapsto F_C (c)]
\]

The functional dependency of the class \texttt{C} gives rise to a type family
$F_C$ which can be used to refer to the type variable $d$ using the type
variable $c$. The instance declaration gives rise to an axiom $g$ that states
that $F_C(Int)$ is equal to $Bool$. This could be used to guide type inference, if
the inference algorithm decides that that $a$ is equal to $Int$, it can use the
axiom $g$ to determine that $F_C(c)$, and indirectly $d$, should be equal to
$Bool$. A second use of this relation is to verify if type variables can be
determined by the given type variables and class constraints. One simply needs
to inspect the domain of the substitution for an occurence of this type
variable.

\begin{figure}
\fbox{$\overline{a}; \overline{\pi} \vdash_{D} \theta \rightsquigarrow \theta'$}
\begin{mathpar}
\inferrule*[right=$\text{Step}_D$]
{
    TC \; \overline{u} \in \overline {\pi}
    \\
    TC \; \overline{a} \mid a_{i_1}, \mathellipsis, a_{i_n} \rightarrow a_{i_0}
    \\
    fv(u_{i_0}) \nsubseteq \overline{a} \cup dom(\theta)
    \\
    fv(\overline{u}^{i_n}) \subseteq \overline{a} \cup dom(\theta)
}
{
    \overline{a}; \overline{\pi} \vdash_{D} \theta \rightsquigarrow
    project(u_{i_0},F_{TC_i}(\theta(\overline{u}^{i_n}))) \cdot \theta
}
\end{mathpar}
\caption{Determinacy Relation}
\label{fig:determinacy}
\end{figure}

The determinacy relation $det(\overline{a},\overline{\pi}) = \theta$ is formally
defined as $\overline{a};\overline{\pi} \vdash_{D} \bullet \rightsquigarrow^!
\theta$ where the single determinacy step $\overline{a};\overline{\pi}
\vdash_{D} \theta \rightsquigarrow \theta'$ is defined in
Figure~\ref{fig:determinacy}. The exclamation mark $!$ indicates that we
repeatedly apply the single determinacy step until no additional substitution is
produced for any type class constraint.

For every type class constraint, for every instantiated functional dependency,
if the range of the dependency has not yet been determined but the domain can be
determined, we update the substitution to map the range to the type family
application of the types in the domain of the dependency. Consider the following
example:
\[
\begin{array}{l}
\texttt{class G a b | a -> b}\\
\texttt{class H a b | a -> b}\\
\end{array}
\]

Computing the determinacy substitution $det(a, \{G \; a \; c, H \; c \; b \} )$
would go as follows:
\[
\begin{array}{l@{\hspace{1mm}}l@{\hspace{1mm}}c@{\hspace{1mm}}l}
a; \{\highlight{G \; a \; c}, H \; c \; b\} &\vdash_{D} \bullet
&\rightsquigarrow& [c \mapsto F_G(a)]
\\
a; \{G \; a \; c, \highlight{H \; c \; b}\} &\vdash_{D}  [c \mapsto F_G(a)]
&\rightsquigarrow& [b \mapsto F_H(F_G(a)), c \mapsto F_G(a)]
\\
a; \{G \; a \; c, H \; c \; b\} &\vdash_{D}  [b \mapsto F_H(F_G(a)), c \mapsto
F_G(a)] &\not\rightsquigarrow&
\end{array}
\]

In the first iteration, only the $G \; a \; c$ matches, as $c$ can be determined
by $a$. The $[c \mapsto F_G(a)]$ substitution would then allow the $H \; c \; b$
constraint to be matched, as $c$ is now indirectly determined by $a$ which
allows $b$ to be determined by $c$. All type variables of every range of every
functional dependency are now accounted for and the determinacy relation halts.

Not every set of type class constraints would contain just type variables like
in the previous example. This is what the $project$ function specified in
Figure~\ref{fig:project} is for. To illustrate what it does consider the
following:

$det(a, C \; a \; (Either \;
String \; c))$ can not have $[c \mapsto F_C(a)]$ as a result, $a$ determines
$Either$ $String$ $c$ and not just $c$. If we had a type family $EitherR$
available with the following axiom
\[
  \textbf{axiom} \; eitherR \; a_1 \; a_2 : EitherR(Either \; a_1 \; a_2) \sim a_2
\]
determinacy could instead correctly produce $[c \mapsto EitherR(F_C(a))]$. We
generalize this to $Proj^T_i$, a projection type family for every type
constructor $T$ and parameter with index $i$. These need to be generated for
every data type declaration, for every parameter of the type constructor.

\[
\begin{array}{l l}
  \textbf{type} \; &Proj^T_i(a)
  \\
  \textbf{axiom} \; g \; \overline{a}^n : &Proj^T_i(T \; a_1 \mathellipsis a_n)
  \sim a_i
\end{array}
\]

To really drive the point home, the following more complicated example $det(a, C
\; a \; (Either \; (Maybe \; b) \; c) )$ would produce the following
substitution:
\[
[ b \mapsto Proj^{Maybe}_1(Proj^{Either}_1(F_C(a))), c \mapsto
Proj^{Either}_2(F_C(a))]
\]

\begin{figure}
\fbox{$project(u, \tau_{fam}) = \theta$}
\[
\begin{array}{l c l}
    project(a, \tau) & = & [a \mapsto \tau]
    \\
    project(T \; \overline{u}, \tau) & = &
    \overline{project(u_i, Proj^T_i(\tau))}
    \\
    project(T, \tau) & = & \bullet
\end{array}
\]
\caption{Type Constructor Projection Function}
\label{fig:project}
\end{figure}

Notice that $project$ is a partial function, it only considers type applications
of the form $T \; \overline{u}$ where the leftmost component is a type
constructor. Unfortunately this is not always the case. Consider the following
slightly contrived example:

\begin{verbatim}
    class A a b | a -> b
    class B a b | a -> b
    class C a b | a -> b
    instance (Functor f, A a (f b), B b c) => C a c
\end{verbatim}

The type variable $c$ in the instance declaration of \texttt{C} can be
determined indirectly through $b$. But $b$ would need to be projected out of $(f
\; b)$. We can't use a projection of the form $Proj^T_i$ because $f$ is not a
type constructor but a variable so in this case we would require something more
general.

If our Haskell-like system and our version of \systemfc were to be extended
with kind polymorphism~\cite{yorgey}, we could instead use a version of the
$project$ function that would be more practical and support these cases. This
version of $project$ is specified in Figure~\ref{fig:poly-project}. Instead of
defining projection type families for every type constructor we would only have
to generate exactly two. $L$ and $R$ project the left and right component
respectively of a type application.
%TODO example?

\[
\begin{array}{l c l}
    \textbf{type} \; L &: &\forall k_1 \; k_2. (a : k_1) \rightarrow k_2
    \\
    \textbf{type} \; R &: &\forall k_1 \; k_2. (a : k_1) \rightarrow k_2
    \\
    \\
    \textbf{axiom} \; proj_L &:& L((u_1 : k_2 \rightarrow k_1) \; (u_2 : k_2))
    \sim u_1
    \\
    \textbf{axiom} \; proj_R &:& R((u_1 : k_2 \rightarrow k_1) \; (u_2 : k_2))
    \sim u_2
\end{array}
\]
\begin{figure}
\fbox{$project(u, \tau_{fam}) = \theta$}
\[
\begin{array}{l c l}

    project(a, \tau) & = & [a \mapsto \tau]
    \\
    project(u_1 \; u_2, \tau) & = &
    project(u_1, L(\tau)) \cdot project(u_2, R(\tau))
    \\
    project(T, \tau) & = & \bullet
\end{array}
\]
\caption{Poly-kinded Projection Function}
\label{fig:poly-project}
\end{figure}


\section{Constraint Generation}
\begin{figure}[h]
% TmVar
% TODO include \notin dom
\fbox{$\Gamma \vdash_{tm} e : \tau \rightsquigarrow t \mid \mathcal{C}$}
\begin{mathpar}
\inferrule*[right=TmVar]
{
    (x : \forall \overline{a} \overline{b}. \overline{\pi} \Rightarrow \tau) \in
    \Gamma
    \\
    \overline{\alpha},\overline{d} \; \text{fresh}
    \\
    \theta = [ \overline{\alpha} \mapsto \overline{a}] \; \cdot \; det(\overline{\pi},
    \overline{a})
}
{
    \Gamma \vdash_{tm} x : \theta(\tau) \rightsquigarrow x \; \overline{\alpha}
    \; \theta(\overline{b}) \; \overline{d} \; | \; \overline{d : \theta(\pi)}
}
\\
% TmAbs
\inferrule*[right=TmAbs]
{
    \Gamma, x : \alpha \vdash_{tm} e : \tau \rightsquigarrow t \; | \; \mathcal{C}
    \\
    \alpha \; \text{fresh}
}
{
    \Gamma \vdash_{tm} \lambda x. e : ( \alpha \rightarrow \tau) \rightsquigarrow
    \lambda (x : \alpha) . t \; | \; \mathcal{C}
}
\\
% TmApp
\inferrule*[right=TmApp]
{
    \Gamma \vdash_{tm} e_1 : \tau_1 \; | \; \mathcal{C}_1
    \\
    \Gamma \vdash_{tm} e_2 : \tau_2 \; | \; \mathcal{C}_2
    \\
    \alpha, c \; \text{fresh}
}
{
    \Gamma \vdash_{tm} e_1 \; e_2 : a \rightsquigarrow (t_1 \triangleright c) \; t_2
    \; | \; \mathcal{C}_1, \mathcal{C}_2, c : \tau_1 \sim (\tau_1 \rightarrow \alpha)
}
\\
% TmLet
\inferrule*[right=TmLet]
{
    \Gamma, x : \alpha \vdash_{tm} e_1: \tau_1 \rightsquigarrow t_1 \; | \;
    \mathcal{C}_1
    \\
    \Gamma, x : \alpha \vdash_{tm} e_2: \tau_2 \rightsquigarrow t_2 \; | \;
    \mathcal{C}_2
    \\
    \alpha, c \; \text{fresh}
}
{
    \Gamma \vdash_{tm} (\textbf{let} \; x = e_1 \; \textbf{in} \; e_2) : \tau_2
    \rightsquigarrow (\textbf{let} \; x : \tau_1 = t_1 \; \textbf{in} \; t_2) \;
    | \; \mathcal{C}_1, \mathcal{C}_2, c : \alpha \sim \tau_1
}
\\
% TmCon
\inferrule*[right=TmCon]
{
    (K : \forall \overline{a} . \tau) \in \Gamma
    \\
    \overline{\beta} \; \text{fresh}
}
{
    \Gamma \vdash_{tm} K : [ \overline{\alpha \mapsto \beta}] \tau
    \rightsquigarrow K \; \overline{\beta} \; | \; \bullet
}
\\
% TmCase
\inferrule*[right=TmCase]
{
    \Gamma \vdash_{tm} e_{scr} : \tau_{scr} \rightsquigarrow t_{scr} \mid
    \mathcal{C}_{scr}
    \\
    \alpha, \overline{\beta}, c, \overline{c'} \; \text{fresh}
    \\
    (K_i : \forall \overline{a}. \overline{\tau}^i \rightarrow T \; \overline{a})
    \in \Gamma
    \\
    \Gamma, \overline{x_i : [\overline{a \mapsto \beta}]\tau_{e_i}} \vdash_{tm} e_i
    : \tau_{e_i} \rightsquigarrow t_{e_i} \mid \mathcal{C}_{e_i}
}
{
    \Gamma \vdash_{tm} \textbf{case} \; e_{scr} \; \textbf{of} \; \overline{K \;
    \overline{x} \rightarrow e} : \alpha \rightsquigarrow \textbf{case} \;
    t_{scr} \triangleright c \; \textbf{of} \; \overline{K \; \overline{x}
    \rightarrow t \triangleright c'}
    \\
    \mid \mathcal{C}_{scr}, \overline{\mathcal{C}_e}, c : \tau_{scr} \sim T
    \; \overline{\beta}, \overline{c' : \tau_e \sim \alpha}
}
\end{mathpar}
\label{fig:ct-generation}
\caption{Term Elaboration and Constraint Generation}
\end{figure}
Figure~\ref{fig:ct-generation} describes constraint generation and elaboration
of expressions into System $F_C$. The judgement takes a typing environment and a
expression $e$ and provides a monotype and a set of annotated wanted
constraints.

\section{Match Contexts}
\label{sec:match-contexts}
% TODO rename dictDestruction? no longer defined in terms of dictionary brings
% TODO mention that we can't use superclass projections for coercions
% everything context related in scope
% Explain type class dictionaries
%TODO examples

This procedure deals with bringing relevant structures in scope related
to type class contexts. It brings super classes in scope and type equalities
introduced by functional dependencies.
\begin{verbatim}
class Eq a => Ord a
sort :: Ord a => [a] -> [a]
\end{verbatim}
In this example, \texttt{Eq} is a superclass of \texttt{Ord}. Because having
implemented an instance for \texttt{Eq} is a prerequisite for \texttt{Ord}, we
should read the implication arrow in the class declaration in the other
direction. If we know that there is an instance for \texttt{Ord}, then we can
assume there is one for \texttt{Eq} as well. We could add the \texttt{Ord a =>
Eq a} implication to the program theory as is. But this implication would
always overlap with those from the instance declarations which would make
solving type class constraints non-deterministic.

The solution is to preemptively, additionally add the instantiated super class
constraints from the context as given class constraints. And we do the same with
the superclasses of the superclasses effectively computing the transitive
closure of the superclass relation. This implies that this procedure can only
terminate when the superclass relation is represented by a directed acyclic
graph.

In the presense of functional dependencies type class contexts also bring type
equalities into scope for each functional dependency of the class.
\begin{verbatim}
class C a b | a -> b
f :: C Int b => b -> Bool
\end{verbatim}
In this example, the constraint \texttt{C Int b} and the functional dependency
of the class \texttt{C} would give rise to the given equality constraint
\texttt{$F_C$(Int) $\sim$ b}.

Like most procedures in this chapter, it's result has a System $F_C$
counterpart. It generates Match Contexts, which are case expressions with a
single match and a hole.
\begin{figure}[h]
$$
\mathbb{E} ::= \square \mid \textbf{case} \; d \; \textbf{of} \; K \;
(\overline{b : k}) \; (\overline{c : \psi}) \; (\overline{x : v}) \; \rightarrow
\mathbb{E}
$$
\end{figure}
We denote match contexts with $\mathbb{E}$ and $\mathbb{E}[t]$ is the match
context with the hole replaced with the term $t$.

For every class constraint, we recursively pattern match on each corresponding
type class dictionary, bringing the super class dictionaries and FD-induced
coercions and existential type variables in scope.

Lastly, it provides an extended typing environment containing the exposed type
variables and type class methods.

\begin{figure}
\fbox{$\Gamma \vdash_{\mathbb{E}} \mathcal{P} \rightsquigarrow \mathbb{E}
       \mid \mathcal{C}_g \mid \Gamma$}
\begin{mathpar}
\inferrule*[right=Hole]
{
}
{
    \Gamma \vdash_{\mathbb{E}} \bullet \rightsquigarrow \square \mid \bullet
    \mid \Gamma
}
\\
% TODO kind b'
\inferrule*[right=MCtx]
{
    \textbf{class} \; \forall \overline{a} \overline{b}. \overline{\pi}
    \Rightarrow TC \; \overline{a} \mid \overline{fd} \; \textbf{where} \; f ::
    \sigma
    \\
    fd_i \equiv \overline{a}^{i_n} \rightarrow a_{i_0}
    \\
    \overline{d}, \overline{c}, \overline{b'}, f' \; \text{fresh}
    \\
    \overline{\Gamma \vdash_{cc} \pi \rightsquigarrow \tau}
    \\
    \Gamma \vdash_{ty} \sigma \rightsquigarrow v
    \\
    \phi_i = F_{TC_i}(\overline{a}^{i_n}) \sim a_{i_0}
    \\
    \overline{\Gamma \vdash_{eq} \phi \rightsquigarrow \psi}
    \\
    \theta = [ \overline{a} \mapsto \overline{u}, \overline{b} \mapsto
    \overline{b'}]
    \\
    \Gamma,f': \theta(\sigma), \overline{b'} \vdash_{\mathbb{E}} \overline{d : \theta(\pi)}, \mathcal{P}
    \rightsquigarrow \mathbb{E}' \mid \mathcal{C}_g' \mid \Gamma'
    \\
    \mathbb{E} = \textbf{case} \; d \; \textbf{of} \; K_{TC} \; \overline{b'} \;
    (\overline{c : \theta(\psi)}) \; (\overline{d : \theta(\tau)}) \; (f' :
    \theta(v)) \rightarrow \mathbb{E}'
}
{
    \Gamma \vdash_{\mathbb{E}} (d : TC \; \overline{u}),\mathcal{P}
    \rightsquigarrow \mathbb{E} \mid \overline{c : \theta(\phi)}, \overline{d :
    \theta(\pi)}, \mathcal{C}_g' \mid \Gamma'
}
\end{mathpar}
\caption{Match Contexts}
\end{figure}
\section{Class Elaboration}
%TODO refer to \ref{fig:class}
Because type classes are represented by dictionaries in System $F_C$ at runtime,
a class declaration gives rise to an $F_C$ data declaration for this dictionary.
The type of the data constructor reflects what will be stored in this
dictionary.  As expected it stores the instance method with type $v$ and the
super class dictionaries of types $\overline{\tau}$.

The method type of specified in the declaration does not entirely correspond
with the actual method type. The resulting type is modified to include the type
class of the declaration as a class constraint.

The method implementation simply matches on the type class dictionary to extract
the actual method within.

There are two interesting additions. The first is that for each functional
dependency $fd_i$ , a corresponding type family declaration $F_{TC_i}$ is
generated as well as a type equality that maps the domain of the functional
dependency to the range. The latter is used as the type for the coercions stored
in the dictionary.

Second, the class declaration can contain existential type variables denoted by
$\overline{b}$. These are type variables that appear the context
$\overline{\pi}$ but not in the type class parameters $\overline{a}$. With
functional dependencies these existential variables are not necessarily
ambiguous. These existential types are put in the type class dictionary as well.
More on ambiguousness is explained in chapter~\ref{cha:conditions}.

Notice that, unlike many previous formalizations of type classes, type class
declarations to not extend the program theory for the reasons explained in
section \ref{sec:match-contexts}.
\begin{figure}
\fbox{$\Gamma \vdash_{cls} cls \rightsquigarrow \overline{decl} \mid
       \Gamma_c$}
\begin{mathpar}
% TODO kinds
% TODO too huge
% TODO conditions as seperate rules
% TODO type families should end up in the environment?
\inferrule*[right=Class]
{
    \overline{c}, \overline{d}, x \; \text{fresh}
    \\
    \Gamma, \overline{a} \vdash_{ty} \sigma \rightsquigarrow v
    \\
    \overline{\Gamma, \overline{a}, \overline{b} \vdash_{cc} \pi
    \rightsquigarrow \tau}
    \\
    fd_i \equiv \overline{a}^{i_n} \rightarrow a_{i_0}
    \\
    \psi_i = F_{TC_i}(\overline{a}^{i_n}) \sim a_{i_0}
    \\
    \sigma \equiv \forall \overline{a'}. \overline{\pi'} \Rightarrow \tau'
    \\
    \sigma_{real} = \forall \overline{a} \overline{a'}. \; TC \; \overline{a} \Rightarrow
    \overline{\pi'} \Rightarrow \tau'
    \\
    \Gamma \vdash_{ty} \sigma_{real} \rightsquigarrow v_{real}
    \\
    t_f = \Lambda \overline{a} \overline{a'}. \lambda(d : T_{TC} \; \overline{a}). \textbf{case}
    \; d \; \textbf{of} \; K_{TC} \; \overline{b} \; (\overline{c : \psi}) \;
    (\overline{d : \pi}) \; (x : v) \rightarrow x \; \overline{a'}
    \\
    \overline{decl_c} =
    [\textbf{data} \; T_{TC} \; \overline{a} \;
    \textbf{where} \; K_{TC} \colon \forall \overline{a} \overline{b}. \;
    \overline{\psi} \Rightarrow \overline{\tau} \rightarrow v \rightarrow T_{TC}
    \; \overline{a}
    , \overline{\textbf{type} \, F_{TC_i} \; \overline{a}^{i_n}}^m
    , \textbf{let} \; f : v_{real} = t_f
    ]
}
{
    \Gamma \vdash_{cls} \textbf{class} \; \forall \overline{a} \overline{b} .
    \overline{\pi} \rightarrow TC \; \overline{a} \mid \overline{fd}^m
    \textbf{where} \; f :: \sigma \rightsquigarrow \overline{decl_c} \mid [ f :
    \sigma_{real} ]
}
\end{mathpar}
\caption{Class Elaboration}
\label{fig:class}
\end{figure}

\section{Instance Elaboration}
Most of the heaving lifting and FD-related things are handled by match
contexts~\ref{sec:conditions} and the superclass
entailment~\ref{fig:superclass-entailment} and axiom
generation~\ref{fig:axiomgen} rules.

Note that the $tm$ judgement has been given a type signature instead of
it being the result of the judgement. This is handled by the subsumption
rule~\ref{fig:subsumption}.
\subsection{Axiom Generation}
As mentioned in section~\ref{sec:determinacy}, every instance declaration gives rise
to top-level type family axioms for each functional dependency.
% TODO liberal coverage condition with conditions?
\subsection{Superclass Entailment}


\subsection{Type Subsumption}
%TODO untouchables \overline{a} passed to entailment
%TODO quote subsumption?
For type checking expressions which are explicitly typed, we employ a procedure
called \textit{type subsumption}. We say that a polytype $\sigma_1$
\textit{subsumes} $\sigma_2$ if any expression of type $\sigma_1$ can also be
assigned the type $\sigma_2$. This concept is also found in systems with
subtyping, like object-oriented languages, where we can instead assign a
supertype of the expressions actual type. In Haskell however, this means that
the type $\sigma_2$ is less general or polymorphic than $\sigma_1$.
%TODO outsidein(x) definition 3.3

In practice, this means that we elaborate the term and infer a type, add a
constraint that says that the monotype component in the provided type signature
should be equal to the inferred type, and add the class constraints in the
signature as given class constraints before constraint entailment. Or in other
words, the term should have the monotype in the signature under the assumption
that the class constraints are satisfied.

In addition to the class constraints in the signature, we also add the
transitive closure of the superclass relation and the FD-induced type equalities
using match contexts.
\begin{figure}
    \fbox{$P; \Gamma \vdash_{inst} inst \rightsquigarrow \overline{decl} \mid
    P_i$}
\begin{mathpar}
\inferrule*[right=Instance]
{
    d_I, \overline{d} \; \text{fresh}
    \\
    \Gamma \vdash_{\mathbb{E}} (\overline{d : \pi}) \rightsquigarrow \mathbb{E}
    \mid P_{\mathbb{E}} \mid \Gamma_{\mathbb{E}}
    \\
    P_I = P, P_{ax}, \overline{d : \pi}, P_{\mathbb{E}}
    \\
    \Gamma_I = \Gamma_{\mathbb{E}}, \overline{a}, \overline{b}
    \\
    S_I = \forall \overline{a} \overline{b}. \; \overline{\pi} \Rightarrow TC \;
    \overline{u}
    \\
    \overline{\Gamma, \overline{a}, \overline{b} \vdash_{cc} \pi
    \rightsquigarrow \tau}
    \\
    (f : \forall \overline{a'} \overline{b'}. \; TC \; \overline{a'}
    \Rightarrow \overline{\pi'} \Rightarrow \tau) \in \Gamma
    \\
    P_I, d_I : S_I; \Gamma_I \vdash_{tm} e : [\overline{a'} \mapsto
    \overline{u}] (\forall \overline{b'}. \; \overline{\pi'} \Rightarrow \tau)
    \rightsquigarrow t
    \\
    S_I \hookrightarrow P_{ax}
    \\
    P_I \vdash_{sc} TC \; \overline{u} \rightsquigarrow
    (\overline{\tau_b}, \overline{\gamma_c}, \overline{t_d})
}
{
    P; \Gamma \vdash_{inst} \textbf{instance} \; \forall \overline{a}
    \overline{b}. \overline{\pi} \Rightarrow TC \; \overline{u} \;
    \textbf{where} \; f = e \\
    \rightsquigarrow [ \overline{\textbf{axiom} \; P_{ax}} , d_I :: \forall
    \overline{a} \overline{b}. \;  \overline{\tau} \rightarrow T_{TC} \;
    \overline {u} = \Lambda \overline{a} \overline{b}. \; \lambda\overline{(d :
    \tau)}. \; \mathbb{E}[K_{TC} \; \overline{u} \; \overline{\tau}_b \;
    \overline{\gamma}_c \; \overline{t}_d \; t]] \mid [P_{ax}, d_I : S_I]
}
\end{mathpar}
\caption{Instance Elaboration}
\end{figure}

\begin{figure}
\fbox{$P \vdash_{SC} \pi \rightsquigarrow (\overline{\tau}_b,
    \overline{\gamma}_b, \overline{t}_d)$}
\begin{mathpar}
\inferrule*[right=SC]
{
    \textbf{class} \; \forall \overline{a} \overline{b}. \; \overline{\pi}
    \Rightarrow TC \; \overline{a} \mid \overline{fd}^m
    \\
    \overline{c}, \overline{d} \; \text{fresh}
    \\
    \theta = [\overline{a} \mapsto \overline{u}] \cdot det(\overline{a},
    \overline{\pi})
    \\
    P_{int} \vDash (\overline{d : \theta(\pi)}), (\overline{c :
    \theta(F_{TC_i}(\overline{a}^{i_n}) \sim a_{i_0})}) \rightsquigarrow
    \bullet, \theta_s, \eta_s
}
{
    P_{int} \vdash_{SC} (TC \; \overline{u}) \rightsquigarrow
    (\theta_s(\theta(\overline{b})), \eta_s(\overline{c}), \eta_s(\overline{d}))
}
\end{mathpar}
\caption{Superclass Entailment}
\label{fig:superclass-entailment}
\end{figure}

\begin{figure}
    \fbox{$\overline{S} \hookrightarrow \mathcal{A}$}
% TODO mention or put in rule how to get instatiated fundep types
\begin{mathpar}
\inferrule*[right=AxiomGen]
{
    (fd_i \equiv \overline{a}^{i_n} \rightarrow a_{i_0}) \in (\overline{fd}^m
    \in TC)
    \\
    \overline{g} \; \text{fresh}
    \\
    \theta_i = det(fv(\overline{u}^{i_n}), \overline{\pi})
    \\
    fv(\theta_i(u_{i_0})) \subseteq fv(\overline{u}^{i_n})
}
{
    (\forall \overline{a} \overline{b}. \; \overline{\pi} \Rightarrow TC \;
    \overline{u}) \hookrightarrow \overline{g_i(fv(\overline{u}^{i_n})) :
    F_{TC_i}(\overline{u}^{i_n}) \sim \theta_i(u_{i_0})}^m
}
\end{mathpar}
\caption{Axiom Generation}
\label{fig:axiomgen}
\end{figure}

\begin{figure}
\fbox{$P; \Gamma \vdash_{tm} e : \sigma \rightsquigarrow t$}
\begin{mathpar}
\inferrule*[right=Subsumption]
{
    \Gamma \vdash_{tm} e : \tau_1 \rightsquigarrow t \mid \mathcal{P}; \mathcal{E}
    \\
    \Gamma \vdash_{ty} (\forall \overline{a}. \; \overline{\pi} \Rightarrow
    \tau_2)
    \\
    \overline{\Gamma \vdash_{cc} \pi \rightsquigarrow \tau}
    \\
    c, \overline{d} \; \text{fresh}
    \\
    \Gamma \vdash_{\mathbb{E}} (\overline{d : \pi}) \rightsquigarrow \mathbb{E}
    \mid P_{\mathbb{E}} \mid \Gamma_{\mathbb{E}}
    \\
    P, (\overline{d : \pi}),P_{\mathbb{E}} \vDash \mathcal{P}, \mathcal{E},
    (c : \tau_1 \sim \tau_2) \rightsquigarrow \bullet; \theta; \eta
}
{
    P; \Gamma \vdash_{tm} e : (\forall \overline{a}. \; \overline{\pi}
    \Rightarrow \tau_2) \rightsquigarrow \Lambda \overline{a}. \;
    \lambda(\overline{d : \tau}). \; \mathbb{E}[\theta(\eta(t \triangleright c))]
}
\end{mathpar}
\caption{Type Subsumption}
\label{fig:subsumption}
\end{figure}
