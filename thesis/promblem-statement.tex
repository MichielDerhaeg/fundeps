% vim:ft=tex
\section{Problem Statement}
Despite the popularity of functional dependencies, the feature remains unreliably
implemented in GHC. Due to the lack expressiveness of the previous intermediate
representation based on System F, it's impossible to elaborate (translate)
Haskell programs that are entirely correct (well-typed and perfectly valid
according to the specification) in GHC and are thus rejected by GHC's type checker.

The following example illustrates such a case where GHC would reject a perfectly
valid program.
\begin{verbatim}
    class C a b | a -> b
    instance C Int Bool

    f :: C Int b => b -> Bool
    f x = x
\end{verbatim}

\texttt{f} is a function that simply returns its argument but has a rather
exotic type signature. It contains a polymorphic type \texttt{b} and a type
class constraint for \texttt{C Int b}. Due to the functional dependency in the
class declaration of \texttt{C} we know that \texttt{b} in the type class
constraint could only have one possible type: \texttt{Bool}. This can be
derived from the instance declaration. \texttt{b} should therefore always be
equal to \texttt{Bool}. This means that this program is well-typed and prefectly
valid. Despite that this program is entirely correct, the relationship between
\texttt{Int} and \texttt{Bool} can not be translated into System F.

When support for type families and associated types was added to GHC, the
intermediate representation was extended to support type level functions similar
to type families and explicit type equality coercions. This new representation
was called System $F_C$. Instead of it being another ad-hoc extension, it was a
more general and powerful solution that could host a multitude of existing and
future type system features, e.g. GADTs, while remaining relatively simple
and straightforward to translate to.

The new representation features explicit type equality coercions, proofs that
could be constructed by the compiler, that witness type equalities. These
coercions can be used to explicitly and safely type cast expressions.

The possibility of implementing functional dependencies by elaborating them into
System $F_C$ was not fully explored any further until recently. Karachalias and
Schrijvers\cite{Karachalias:2017:EFD:3156695.3122966} presented a type
inference algorithm for functional dependencies and a way to elaborate them into
System $F_C$.

This thesis will present a fully functional implementation of the aforementioned
algorithm and a description of important implementation specific details and
complications that may arise throughout the development of this implementation.

Eventually, this implementation is not only aiming to implement support for functional dependencies,
but GADTs and type families as well. So the interaction between these features
and their implementations can be studied. Especially conflicts and complications
that may arise during the development of all these features might be very
interesting.

Firstly, this implementation will serve as a validation of the specification
that might reveal potential inaccuracies. And secondly, as a feasibility study
that shows that it is possible to fully implement functional dependencies by
reusing the available framework that is currently in place in GHC.

\begin{verbatim}
    class C a b | a -> b
\end{verbatim}
$\quad\quad\quad\rightsquigarrow$ data $CDict$ $a$ $b$ where

$\quad\quad\quad\quad$ type $FC$ $a$

\begin{verbatim}
    instance C Int Bool
\end{verbatim}
$\quad\quad\quad\rightsquigarrow$ let $d$ :

$\quad\quad\quad\rightsquigarrow$ axiom $fCIntBool$ : $FC$ $Int$ $Bool$

\begin{verbatim}
    f :: C Int b => b -> Bool
    f x = x
\end{verbatim}
$\quad\quad\quad\rightsquigarrow$ let $f$ =
% provide solution maybe entirely complete solution this time
% show problem termination?
% planning

\george{Yes, show and explain the elaboration of the example here, finish it.}


